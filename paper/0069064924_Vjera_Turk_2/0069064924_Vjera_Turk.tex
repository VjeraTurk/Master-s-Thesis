% !TeX encoding = windows-1250
\input{tex_aux/rithesis_preamble}

% pomocu \includeonly moze se kompajlirati samo odredjeno poglavlje, da se skrati vrijeme kompajliranja, dok se ne isprave pogreske u tom poglavlju npr.:

%\includeonly{Poglavlje_1}


\begin{document}

\frontmatter   % - ne dirati

% upisati naziv studija
\degreesubject{Diplomski studij ra�unarstva} % upisati odgovarajuci naziv studija

% upisati vrstu rada
\documenttype{Diplomski rad}  % Zavrsni rad ili Diplomski rad

\title{Metodologija za usporedbu kontekstualiziranih \\ polazi�no-odredi�nih matrica}   % upisati specificni naslov rada

\date{\MONTH~\the\year.}   % ne dirati - mjesec i godina �e se upisati sami

\author{Vjera Turk}  % upisati svoje ime i prezime
\jmbag{0069064924}  % upisati vlastiti JMBAG
\maketitle		% ne dirati

%\makecopyright

% Okruzenje za pisanje posvete. Maknuti komentare ukoliko se �eli napisati posvetu.
%\begin{dedication}
%	Ovo je posveta nekome
%\end{dedication}

\mentor{prof.dr.sc.~Renato Filjar}   % zamijeniti podacima o svojem mentoru
\maketitleabstract

% kreira mjesto za umetnuti stranicu s opisom zadatka - ne dirati
\begin{assignmentpage}
	Naslov zadatka: Metodologija za usporedbu kontekstualiziranih polazi�no-odredi�nih matrica\\	
	Thesis title: Methodology for contextualised origin � destination matrices comparison\\
	Polje znanstvenog podru�ja: Ra�unarstvo\\
	Grana znanstvenog podru�ja: Informacijski sustavi\\
	
	Sadr�aj zadatka:  Polazi�no-odredi�na matrica (POM) omogu�uje sustavnu statisti�ku procjenu migracija stanovni�tva u zadanom prostorno-vremenskom okviru. Za razliku od tradicionalnog pristupa brojanja putovanja i putnika, za procjenu POM-e danas se sve vi�e koristi statisti�ka analiza podataka iz suvremenih informacijskih i komunikacijskih sustava (zapisi o aktivnostima u javnoj pokretnoj mre�i, zdru�ena o�itanja prijamnika za satelitsku navigaciju i sl.), �ime je omogu�eno pobolj�anje kvalitete procjene preslikavanjem POM-e na kontekst. \textbf{Pojavljuje se potreba za objektivnom procjenom kvalitete POM-e u odnosu na referentnu (kontrolnu). U ovom radu potrebno je definirati odnosne parametre kvalitete POM-e te razviti metodologiju usporedbe dviju POM-a dobivenih razli�itim postupcima procjene i s podatcima iz razli�itih izvora}. Usporedbu obaviti kori�tenjem numeri�kog i grafi�kog oblika POM-e. Metodologiju je potrebno izvesti u programskom okru�enju za statisti�ko ra�unarstvo R te demonstrirati njenu primjenu na slu�aju usporedbe dviju POM-a. Komentirati dobivene rezultate sa stajali�ta apsolutne i relativne to�nosti matrica.
\end{assignmentpage}

% kreira mjesto za umetnuti stranicu s izjavom o samostalnoj izradbi zadatka - ne dirati
\begin{honestystatementpage}
	\input{Izjava}
\end{honestystatementpage}

% Okruzenje za pisanje zahvale

%\begin{acknowledgments} % staviti znak komentara ukoliko se ne stavlja tekst zahvale
%	% !TeX encoding = windows-1250
\vspace{5pt}

\begin{flushleft}
\noindent


\end{flushleft}  
%\end{acknowledgments}

% kreiranje popisa sadrzaja, slika i tabela - ni�ta ne dirati
\tableofcontents
\listoffigures
\listoftables

\mainmatter		% ne dirati

% Ovdje pomo�u include funkcije ucitavati kreirana poglavlja. Poglavljima dajte logicna imena s obzirom na sadrzaj prikazan u njima (bez razmaka u imenu).
%\include{Intro}   % ovo poslije staviti pod komentar kada se nau�i koristiti

% !TeX encoding = windows-1250
\chapter{Uvod}

Polazi�no-odredi�na matrica (POM) alat je koji omogu�uje sustavnu statisti�ku procjenu migracija stanovni�tva na nekom podru�ju u zadanom prostorno-vremenskom okviru. Za razliku od tradicionalnog pristupa brojanja putovanja i putnika, za procjenu POM-e danas se sve vi�e koristi statisti�ka analiza podataka iz suvremenih informacijskih i komunikacijskih sustava (zapisi o aktivnostima u javnoj pokretnoj mre�i, zdru�ena o�itanja prijamnika za satelitsku navigaciju i sl.), �ime je omogu�eno pobolj�anje kvalitete procjene preslikavanjem POM-e na kontekst. 

Pojavljuje se potreba za objektivnom procjenom kvalitete POM-e u odnosu na referentnu (kontrolnu). U ovom radu definirani su odnosni parametri kvalitete POM-e te je razvijena metodologija usporedbe dviju POM-a dobivenih razli�itim postupcima procjene i s podatcima iz razli�itih izvora. Usporedba je obavljena kori�tenjem numeri�kog i grafi�kog oblika POM-e. 

Metodologija je izvedena u programskom okru�enju za statisti�ko ra�unarstvo R te je demonstrirana njena primjena na slu�aju usporedbe dviju POM-a. Dobiveni rezultati komentirani su sa stajali�ta apsolutne i relativne to�nosti matrica.
\cite{Peterson:2007.}
\cite{Alhazzani:2016.}
\cite{Bahoken:2013.}
\cite{Filic:2016.}
\chapter{Polazi�no-odredi�na matrica}

 Polazi�no-odredi�ne matrice (POM) sadr�e broj putovanja izme�u svakog para polo�aja unutar nekog podru�ja za odre�en vremenski okvir. 
  
 The Origin-Destination matrices (ODs) provides the number of trips between each pair of locations in the area for a specified time window 
 
 ...which specify the travel demands between the origin and
 destination nodes in the network. 
 
Tranzitna, t-POM
koncentracija u radu na CDR (?)
            
\section{Tradicionalni pristupi generiranju POM-a}

 Traditional methods include running surveys within cities and estimating the flows between locations of the city from the feedback of those surveys. Such methods consume longer periods of time and are inaccurate at times. They usually span smaller population sample sizes and thus are more prone to biases.

Recent research in the domain of ubiquitous computing provided alternative methodologies for estimating more accurate ODs from user generated datasets like cell phone data. 

\subsection{Ankete}
cijena anketiranja  (u jednom od radova 10 eura po ispitaniku?) \gls{rsi}
\subsection{Prebrojavanje vozila}
ru�no, video
\subsection{Modeliranje toka}

Mathematical modeling of traffic requires a lot of data and other information about the road network and the travel demand. 
The accuracy of the modeled traffic situation depends on the quality of the available information, and how this data is combined and weighted from different sources. The
travel demand is a key component and nearly every traffic model requires a tableau OD/trip matrix/table specifying the travel demand between different places in the network. Such a tableau is called an Origin�Destination matrix, or OD-matrix for short; synonymously used terms are trip table or (origin�destination) trip matrix. 

\subsection{Problematika i ograni�enja tradicionalnih na�ina}
Zastarijevanje 

\section{POM iz zapisa o aktivnostima u javnoj pokretnoj mre�i}

Today, with the ubiquity and pervasiveness of technology, data generated from mobile phones enable data analysts to better understand the behavior of individuals across many dimensions including their mobility patterns. 

Unlike traditional analyses, the nature of this data mining approach forces us to first provide rigid, formal definitions of exactly what we mean by the terms origin, destination and journey. Here, both origins and destinations are subcategories of the overarching concept of a �stop�. A stop is defined as a set of contiguous network events that occur at the same location, over a minimum period of time. 

This notion is parameterized to ensure we have sufficient confidence that any stop we have detected is not a transient location, but actually a location that the individual has actually settled in.

An algorithm must consequently be developed to exhaustively mine each person�s event series for such stops. Once achieved, the algorithm must next detect pairs of contiguous stops which occur at different locations and hence reveal movement. This pair can then be designated as a journey -
the initiating stop becoming the journey�s origin, and the concluding stop as the journey�s destination. 


coordination with traditional tehniques is key to providing optimal solution in future.\cite{Goulding:2016.}

\gls{gsm}

\subsection{Razlike u pristupima}
        Tranzitna t-POM,

		Definiranje putovanja
	        Period - Departure/Arrival time
        "Flow between O-D pair w that departed its origin during time interval k"
        \cite{Bera:2011.}
       
        Grad, Dr�ava
        CDR- POM u zemljama u razvoju
\subsection{To�nost polo�aja}
 to�nost polo�aja - aproksimacija s BS ili signal strenght (RSSI)
 + neka ona drugo baza LTO(?)

\subsection{Geometrija prostorne podjele}

\subsubsection{Heksagoni}
\subsubsection{Voronoi tesalacije}
\subsubsection{Jedinice samouprave}
\subsubsection{Pravokutna mre�a}

\subsection {Dobre prakse u generiranju POM iz CDR}
Skaliranje CDR POM (primjerak $->$ pouplacija)
(linking to transport infrastructure?)
k-anonymization

\newpage
\section{Drugi primjeri automatskog prikupljanja}

\subsection{Zdru�ena o�itanja prijamnika za satelitsku navigacij}
\subsection{Javni prijevoz i \textit{pametne kartice }(\textit{Smart Card }sustavi)}
82\% putovanja javnim prijevozom naprave korisnici javnog prijevoza sa pametnim karticama.  \cite{Travassoli:2016.}

\vspace{10pt} 
% !TeX encoding = windows-1250

\chapter{Moj pogled na kvalitetu \glstext{pom} temeljem parametara}

\todo[inline]{motivacija, teza, formulacija problema}
\section{Motivacija}
%\subsubsection{Problematika i ograni�enja tradicionalnih pristupa}
�irenje gradova i rast stanovni�tva rezultirali su rastu�im brojem sve ozbiljnijih prometnih zastoja u gradovima diljem svijeta. Prepoznata je potreba za strategijama upravljanja prometom i gradom op�enito koje �e uva�iti dinamiku razvoja stanovni�tva i njegove migracije u kontekstu suvremenih dru�tvenih i gospodarskih kretanja. %kako bi se suo�ili s izazovima koje donosi brzo razvijaju�a okolina i demografija populacije. 
Djelotvorno upravljanje i kontrola prometa doprinose pove�anju sigurnosti, kvalitete prometne i kvalitete transportno-logisti�ke usluge, poti�e ekonomski rast i smanjuje zaga�enje okoli�a. %Zbog dinamike kojom se gradovi mijenjaju razumno je pitati se ima li smisla koristiti \gls{pom}-e stare vi�e godina za modeliranje dana�njeg prometa.
Suvremeni odnosi promijenili su prirodu kretanja ljudi i dobara zbog �ega tradicionalni pristup postaje zastario i neprimjeren. Konkretno, \textbf{strate�ka planiranja izgradnje prometne infrastrukture, koja se vode vremenski nezavisnim modeliranjem potra�nje, trebala bi uzeti u obzir vremenski zavisne modela}. Osobitu pa�nju treba posvetiti prometnoj potra�nji tijekom dnevnih vrhunaca i pokazateljima kako �e se taj vrhunac mijenjati u bliskoj i dalekoj budu�nosti. \textbf{Tradicionalni modeli procjene pretpostavljaju da se �irina vremenskog vrhunca u budu�nosti ne�e mijenjati iako se ona ve� sad razlikuje od grada do grada (Slike \ref{fig:peak})} \cite{Thakur:2018.}.

Kako prikazati kvalitetu \gls{pom}-e bez oslanjanja na \textit{grand turth} koja porastom zna�ajnijih i br�ih promjena pove�ava utjecaj pristranosti kretanju u pro�losti? Postoji potreba za objektivnom procjenom kvalitete \gls{pom}-a definiranjem odnosnih parametara kvalitete.

\gls{cdr} se isti�e kao izvor gotovo jednako zastupljen u zemljama u razvoju i razvijenim zemljama, s penetracijom oko 40\%-50\% stanovni�tva za pojedine operatore \cite{Bonnel:2015.}. U Francuskoj je 2008. 80\% stanovni�tva starijeg od 12 godina posjedovalo mobilni ure�aj \cite{Bonnel:2015.}a 2016. �ile je imao �ak 132 mobilne pretplate na 100 stanovnika \cite{Graells-Garrido:2016.}. Praksu odbacivanja ogromnog dijela uzorka zbog niske potpunosti putanja treba zamijeniti rekonstrukcijom putanja.

%Tradicionalni pristup validaciji procjenjuju sli�nost s \gls{pom} koja je nu�no nastala u pro�losti, pa se izmjene u odnosu na prethodnu matricu tuma�e kao "gre�ka" iako se lako zapitati je li zapravo nova matrica ona koja opisuje stvarnost bolje od referentne.

%***strate�ko planiranje izgradnje prometne infrastrukture i urbanizacija moraju uzeti u obzir dinamiku promjena - �lanak gdje se obja�njava da je pretpostavka kod strate�kog planiranja da �e peak-hour ostati isti, to su projektiranja koja moraju razmi�ljati unaprijed desetcima godina (a to znaci uzeti u obzir mnog faktora ) https://home.kpmg/au/en/home/insights/2018/07/avoid-peak-hour-new-transport-models.html ***

\begin{figure}[!htpb]
	\begin{center}
		\subfloat[\textit{�irina} jutarnjeg vrhunca prometne potra�nje u 3 grada]{\label{fig:peakcity}
			\includegraphics[width=12cm,keepaspectratio=true]{peak}}\\ 
		%\hspace{10pt}
		\subfloat[Predikcija �irine vrhunca prometne potra�nje]
		{\label{fig:peak2046}
			\includegraphics[width=12cm,keepaspectratio=true]{2046}}
		\caption{�irina vrhunca prometne potra�nje - vremenski zavisno modeliranje}
		\label{fig:peak}
	\end{center}
\end{figure}

\textbf{Istra�iva�i pro�iruju podru�je primjene zaklju�aka analize kretanja stanovni�tva uzimaju�i u obzir njihov kontekst \cite{Barbour:2019.}}. Mogu�nost preciziranja odredi�ta s razine zone na razinu zgrade i odre�ivanje svrhe (motiva) kretanja trebalo bi obuhvatiti procjenom kvalitete. (...)

\begin{comment}
TODO:Vrati se tu, ok odlomak.
Novi moderni postupci procjene \gls{pom}-a postaju sve vi�e razmatrana opcija s obzirom na fleksibilnost, cijenu i a�urnosti koju nude u odnosu na tradicionalne.  Prepoznat je potencijal, ne samo za analizu urbanih i dr�avnih razmjera, ve� i za analizu mobilnosti planetarnog razmjera dijeljenjem i kombiniranjem razli�itih izvora podataka koji nose podatak o polo�aju \cite{Scepanovic:2015.}\cite{Hui:2010.}. 
\gls{cdr} se isti�e kao izvor gotovo jednako zastupljen u zemljama u razvoju i razvijenim zemljama, s penetracijom oko 40\%-50\% stanovni�tva za pojedine operatore \cite{Bonnel:2015.}. U Francuskoj 2008. 80\% stanovni�tva starijeg od 12 godina posjedovalo je mobilni ure�aj \cite{Bonnel:2015.}. Istra�ivanje iz 2016. provedeno na podacima u �ileu  iznosi podatak da �ile ima �ak 132 mobilne pretplate na 100 stanovnika \cite{Graells-Garrido:2016.}. 
\end{comment}
\begin{comment}
\todo[inline]{Nije ba� tako. Brkaju se znanost, istra�ivanje, tehnologija i kori�tenje. Naru�itelj ho�e usporediti s onim �to je njemu poznati, ali to je usporedba kru�aka i jabuka.
Gledano iz perspektive validacije istra�iva�kog modela, usporedba se mo�e raditi s drugom, neovisnom metodom koja daje pribli�no jednako kvalitetne rezultate.

A problem ostaje otvoren za raspravu, \textbf{koliki je udio populacije stvarno pokriven i je li model razvije na takvom podskupu populacije relevantan.} }

\todo[inline]{Volio bih vidjeti konkretniji prikaz koji �e ukazati na su�tinske prednosti i nedostatke, ne investicijske. Naime, cilj ovog poglavlja jest ukazati kako \textbf{suvremeni tehnolo�ki sustavi nedostatke tradicionalnih pristupa uspje�no prevladavaju.}
}
\end{comment}
\section{Pregled prethodnih istra�ivanja}
\subsection{Primjeri kori�tenja vanjskih izvora konteksta u analizi kretanja}

\subsubsection{Studija A - }
U jednom od svojih radova grupa autora kvantizira vezu izme�u ukupno 23 namjenske kategorije \gls{pois} iz slu�bene gradske baze (npr. tvornice, bolnice, javne �kole, religijski objekti, hoteli, knji�nice, sportski objekti) i onime �to nazivaju 3 tipa privla�enja. Dolaze do zaklju�ka da postoje 3 tipa privla�enja te da svaki ima karakteristi�an tok (gusto�u ukupnog toka, udaljenost i disperziju polazi�ta tokova usmjerenih prema objektu tog tipa privla�enja). Svaku od kategorija karakterizira jedan od ta 3 tipa privla�enja. \cite{Alhazzani:2016.}  U radu koriste podatke iz slu�bene baze s 12,000 \gls{pois} i \gls{cdr} zapisima iz razdoblja od mjesec dana. Cilj projekta bio je ispitati mobilnost na podru�ju grada Riyadha, u Saudijskoj Arabiji za planiranje izgradnje podzemne �eljeznice.  Njihovi rezultati mogu se primijeniti u planiranju pozicioniranja budu�ih objekata u gradu.   
% We used a statistical significance testing approach to rigorously quantify the relationship between Points of Interests (POIs) types (services) and the 3 patterns of Urban Attractors we detected. \gls{pois} Points of Interes (Urban Attractors ima 22 kategorije koristi bazu s POI)
\subsubsection{Studija B - Model Raspodjele Toka}
Inovativni pristup kontekstualizaciji toka kori�tenjem OpenStreetMap baze predstavljen je u radu \cite{Stupar:2018.}. Na osnovu pripadaju�ih opisnih podataka, prostorni objekti na podru�ju interesa kategorizirani su prema tipu socio-ekonomske aktivnosti u koju su uklju�eni. Definirano je ukupno 6 kategorija:
Dom (Home), Posao (Work), Zdravlje (Health), Edukacija (Education), Zabava (Leisure) i Ostalo (Other). Razvijen je vjerojatnosni model koji, na osnovu broja objekata pojedine kategorije u odredi�noj �eliji i promatranog vremenskog okvira, ukupni ulazni tok u odredi�nu �eliju dijeli na 6 tokova usmjerenih prema objektima tih kategorija. Iz jedne \gls{pom}-e tako  se dobije 6 \gls{pom}-a, po jedna za svaku od kategorija. U demonstraciji modela kori�tene su  \gls{pom}-e dobivene iz javno dostupnih, anonimiziranih telekomunikacijskih zapisa na podru�ju kineskog grada Shenzhena.\cite{Filic:2016.}\cite{data:2019.}
\textbf{Definiranjem konteksta putovanja na ovaj na�in sa�uvana je mogu�nost postizanja dobre rezolucije u vremenu, a istovremeno je dan kontekst koji uklju�uje tipi�ne komutacijske \textit{Home} i \textit{Work} kategorije, te umjesto op�enite kategorije \textit{Other} precizira 4 kategorije}. 

Definiranjem konteksta dobivamo nove informacije o kretanju unutar �elije, odnosno o kona�noj destinaciji nakon ulaska u �eliju. U tom smislu poznata je informacija o internom kretanju.

\subsubsection{Studija C - Planiranje odr�ivih gradova}
(...)
\cite{Barbour:2019.}
\subsubsection{Studija D - \cite{Chen:2019.}}
\label{Chen}

\section{Teza}
\todo[inline]{ Statisti�ka analiza podataka iz suvremenih informacijskih i kounikacijskih sustava sve vi�e se koristi za procjenu \gls{pom}-a �ime je omogu�eno pobolj�anje kvalitete procjene preslikavanjem na kontekst.\\
	Odre�ivanje konteksta pro�iruje podru�je pokrivanja u fizi�kom i informacijskom smislu te pove�ava to�nost pome unosom relevantnih atributa vezanih za na�in i svrhu putovanja - �ime je putovanje to�nije opisano\\
	- izbor metode procjene definira preciznost odre�ivanja konteksta}


\subsection{Redefiniranje validacije \glstext{pom}}
%moj pogled na validaciju pom
...
\textit{Validacija \gls{pom}-e je proces koji ima za cilj procjenu obilje�ja kvalitete te \gls{pom}-e i zna�i identifikaciju obilje�ja kvalitete \gls{pom}-e, njenog sadr�aja, njene smislenosti, a izra�ava se slijede�im  indikatorima kao �to su: to�nost, prostorno obuhva�anje(?), zrnatost, definicija putovanja, prostorna rezolucija, vremenska rezolucija, �irina toka i gusto�a informacija.}
%\todo[inline]{- pojava novih izvora podataka i postupaka procjene migracija\\- ve�a preciznost odre�ivanja polo�aja. polazi�ta i odredi�ta\\- kontekstualizacija\\- to�nost opisa putovanja raste\\ - podru�je primjene raste\\- potencijalno vrlo velik reprezentativni uzorak (CDR)\\- jako dobra rezolucija u vremenu\\- osigurati anonimnost}



% !TeX encoding = windows-1250
\chapter{Metodologija}
\label{met}
%\todo[inline]{opis sklopa/programske podr�ke. Opis rje�avanja zadatka i odgovaraju�i prora�uni}

%Studying the behaviour of moving objects over time and their interaction, either between objects or with environment, plays a crucial role in understanding how they use space and more importantly how they interact each other. Moving objects are moving within a particular area over time, thus a snapshot of a trajectory pattern might be seen as a spatial point pattern. This aspect then empowers us to study the behaviour of moving objects within space and over time. A set of locations, usually non-uniformly distributed within a certain region, can be considered as a realisation of a spatial point process.

\section{Uvjeti usporedbe}
\gls{pom}-e koje se uspore�uju trebaju ispunjavati uvjete usporedbe, odnosno imati slijede�a zajedni�ka obilje�ja:

\begin{itemize}
	\item  isto podru�je/grad/dr�ava
	\item  isto doba godine

\begin{comment}
	\item vremensko razdoblje $\mathcal{T}$ iste duljine\\ 
Prednost novih postupaka procjene \gls{pom} iz \gls{cdr} u odnosu na ankete stanovni�tva je da mogu uhvatiti tjedne i sezonske uzorke \cite{Calabrese:2011.}
%Compared with traditional census data, our methodology to detect ODMs from mobile phone traces has several advantages: It can capture the weekday and weekend patterns as well as seasonal variations. It can capture work trips and non-work trips, which is essential for trip chaining and activity based modeling. Zaklju�ak 
%Istra�ivanje iz 2013. \cite{Varun:2013.} pokazalo je da su u Obali Bjelokosti najve�a odstupanja u gusto�i poziva u prvom i posljednjem mjesecu godine, gdje je 2. tjedan u prosincu bila najve�a, a sredinom sije�nja najni�a zabilje�ena gusto�a (u periodu promatranja od prosinca do travnja). 
\end{comment}

Zabilje�enu varijaciju po tjednima u jednoj godini iz istra�ivanja procjene  \gls{pom}-a iz podataka o javnom prijevozu na podru�ju Nizozemske mo�emo vidjeti na slici \ref{fig:per_week}. Prednost novih postupaka procjene \gls{pom} (npr. iz \gls{cdr}) nad procjenama iz anketa stanovni�tva je da mogu uhvatiti tjedne i sezonske uzorke \cite{Calabrese:2011.}.


%From Figure 8 if we compare the call density across different months, and days of the month, it is interesting to note that the call density towards the middle of the month is lower in January than in other months. On the same lines note that call density in the second week of December is higher than other months, but reduces in the following weeks. Based on 2These are not recommendations, but only observations based on visual inspection this observation, we are attempting to find the mobility pattern of mobile phone users and to determine whether the mobility pattern influences this call density pattern. \cite{Varun:2013.}

\begin{figure}[!htbp]
	\begin{center}
		\includegraphics[width=15cm,keepaspectratio=true]{per_week}
		\caption{Broj putovanja javnim prijevozom zabilje�enih \textit{OV-chipcaart} pametnim karticama u Nizozemskoj po tjednima u jednoj godini. Zelena linija ozna�ava prosje�nu vrijednost broja putovanja radnog tjedna \cite{Kuhlm:2015.}}
		\label{fig:per_week}
	\end{center}
\end{figure}
	\item  ako se radi o jednom danu - idealno isti dan u tjednu

Rezultati nacionalne ankete u Sjedinjenim Ameri�kim Dr�avama pokazuju da je prosje�an broj dnevnih putovanja po stanovniku 4.18 radnog, te 3.86 neradnog dana, �to je u skladu s zaklju�cima suvremenih postupaka procjene.\cite{Calabrese:2011.}\cite{Bahoken:2013.}\\

Osim o�ekivane razlike izme�u radnog i neradnog dana, uo�eno je da je petak (zadnji radni dan u tjednu), druga�iji od ostalih radnih dana \cite{Calabrese:2011.}\cite{Scepanovic:2015.}\cite{Travassoli:2016.}. 
Uo�ena je poja�ana mobilnost petkom na razini regije (Boston Metropolitan Area) \cite{Calabrese:2011.} i na razini dr�ave (Obale Bjelokosti) s pove�anjem od 35\% u odnosu na nedjelju \cite{Scepanovic:2015.} na temelju \gls{pom}-a izra�enih iz \gls{cdr}. Uo�ena je smanjena mobilnost vezana za javni prijevoz (izvor za procjenu \gls{pom}-a \textit{Pametne kartice za javni prijevoz}) na razini regije (Southeast Queensland, Australija) \cite{Travassoli:2016.}. Petak ima i ne�to druga�iji dnevni uzorak po satima (poja�anje prometa prema kraju radnog vremena zapo�inje ranije) \cite{Travassoli:2016.}. %Petkom je tako�er uo�ena i najvi�a razina telekomunikacijske aktivnosti \cite{Bahoken:2013.} \cite{Bonnel:2015.} �to ukazuje na mogu�u pristranost kada su u pitanju \gls{pom}-e procijenjene iz \gls{cdr}.
	%(...)\\
Dr�avni praznici, elementarne nepogode i druge izvanredne i predvidljive situacije unose odstupanja od ustaljenih uzorka kretanja. Privremene turisti�ke atrakcije i sportski doga�aji tako�er mogu utjecati na odstupanje od uobi�ajenog uzorka kretanja. \cite{Varun:2013.}
\item ista definicija dana

Dio putovanja zapo�inje u jednom, a zavr�ava u drugom danu. Autori \cite{Schneider:2013.}\cite{Toole:2015.} zastupaju definiranje po�etka/zavr�etka dana u 3:00 idu�eg dana, umjesto u 0:00, kako dio putovanja ne bi bio izostavljen zato jer se prote�e kroz 2 dana.
%\cite{Bonnel:2015.} Primjerice odbrojavanje uo�i Nove godine na New Yourk Times Square. 
%Gusto�a poziva na festivalske dane  u prosjeku je 10-15\% ve�a od uobi�ajene. \cite{Varun:2013.}
	
	\item ista kategorija putovanja (tranzitna,  kontekstualizirana, \textit{komutacijska})
	\item obje \gls{pom}-e napravljene objektivno na 
	%cijeloj populaciji ( ***
	reprezentativnom uzorku%)	
\end{itemize}

\section{Usporedba prema odnosnim parametrima \\kvalitete}

Ako se ustanovi da \gls{pom}-e ispunjavaju uvjete usporedbe, uspore�uje ih se prema parametrima definiranim u \ref{param}. Metodologija definira da vrijedi slijede�e:

\begin{itemize}
	\item prostorno obuhva�anje: \textbf{\gls{pom} je bolja ako je prostorno obuhva�anje ve�e} (podru�je neovisno o prometnoj infrastrukturi $>$ podru�je ovisno o prometnoj infrastrukturi)
	\item gusto�a informacija: \textbf{\gls{pom} je bolja �to je gusto�a informacija ve�a (�to je manji broj Nul-�elija)}
	\item ukupna �irina toka: \textbf{\gls{pom} je bolja �to je ukupna �irina toka ve�a}\\
	Podrazumijeva se isti vremenski okvir npr. od 6:00 do 9:00, cijeli jedan dan...
	\item zrnatost:	
	\textbf{\gls{pom} je bolja �to je zrnatost ve�a, pod uvjetom da je dovoljno mala da osigurava k-anonimnost} \\
	K-anonimnost smanjuje rizik da identitet pojedinca bude razaznan od $k-1$ drugih pojedinac (\ref{k-an}). Pove�anjem bilo koje zrnatosti tokovi postaju sve u�i. Uva�avanje k-anonimnosti postavlja ograni�enje na zrnatost \cite{Goulding:2016.}. Tradicionalni postupci procjene \gls{pom} (podatak za Ujedinjeno Kraljevstvo) dozvoljavaju �irinu jednog toka izme�u 12 i 15.
	Agencija za analizu kretanja iz \gls{cdr} Teralytics \cite{Teralytics:2017.} koristi $k$ izme�u 5 i 10.
	\item rezolucija:
	Odlomak \ref{rez} govori o pozitivnoj korelaciji rezolucije i to�nosti. Odbacivanjem to�nost kao jedinog pokazatelja kvalitete pitanje postaje koja rezolucija je optimalna za prikaz dnevnih urbanih migracija u \gls{pom}. Visoka rezolucija podataka omogu�uje visoku zrnatost. Primjerice, ako mo�emo odrediti zavr�etak putovanja u minutu, vremenski okvir teoretski mo�e biti $< 15 min$. Odabir niske zrnatosti umanjuje potrebu za visokom rezolucijom. Primjerice, ako je vremenski okvir �irine 12 sati (2 perioda u danu) nema potrebe da je po�etak i zavr�etak putovanja odre�en u minutu.
	
%Visoka rezolucija pogodovati samo internim putovanjima.	
	%\item Putovanje
\end{itemize}


Optimalne vrijednosti parametara naposljetku �e ovisiti o podru�ju primjene \gls{pom}.  Odre�ivanje optimalnih vrijednosti parametara zahtjeva dodatna istra�ivanja koja bi se nadovezala na metodologiju predstavljenu u ovom radu. 
Primjerice, rezolucija koja dobro odgovara u jednom mo�e biti nepotrebno visoka u drugom podru�ju primjene. Odabir ni�e rezolucije smanjuje koli�inu pohranjenih podataka \cite{Degbelo:2014.} i mo�e smanjiti vrijeme izvo�enja postupka procjene.\\ %***

\subsection{Postupak odlu�ivanja}

Metodologija definira svaki parametar kao jednako vrijedan u dono�enju kona�ne odluke o kvaliteti matrice. POM koja je bolja u vi�e parametara ukupno je kvalitetnija. Parametar koji se ne odnosi ni na jednu matricu se ne uzima u obzir. 

\newpage
%stavljen naglasak za demonstraciju metodologije je analiza kretanja cjelokupnog stanovni�tva grada za strate�ka planiranja na podru�ju grada.\\
%\textit{Ako je podru�je pokrivanja ve�e, matrica je bolja, ako je gusto�a informacija takva i takva, matrica je bolja}\\
%\todo[inline]{\textit{ $+$ Treba raspisati do kraja kako �u odlu�iti je li ne�to bolje po ovom ili onom kriteriju - poglavlje 5 - i tu smo jasni, opet slika ili prebrojavanje - nema drugog. U poglavlju 5. napisat �ete samo da brojite, i gledate slike. A u poglavlju 6 �ete staviti slike i napisati rezultate zbrajanja- ne vi�e od toga.}\\$+$\textit{I nakon toga na primjeru �ilea ili �ega god ho�ete}\textit{Koja ima najvi�e pluseva, ta �e ispasti bolja- sve je jednako vrijedno (svaki parametar?!)}}
%\section{Brzina kretanja}
%\todo[inline]{- Analiza prikladnosti odabira perioda i dimenzije �elija s obzirom na brzinu kretanja agenata ?!\\ - lo�e postavke one s kojima �e putovanja nu�no zavr�iti isklju�ivo u istim/ susjednim �elijama?!- avemove() funkcija iz trajectories?}

%\todo[inline]{$+$ Usporedbu obaviti kori�tenjem numeri�kog i \\ $-$ \textbf{grafi�kog } oblika POM-e}

%\todo[inline]{$-$ Kontekstualizacija i interna putovanja- * kontekstualizacija donosi da znamo interna kretanja/ putovanja izme�u objekata \ref{in}}

%\todo[inline]{$+$\textit{Mogli biste uz definiciju ovih kriterija, samo sliku ubacite iz ovih nekih simulacija, Vara�dina, Koprivnice, kao ilustraciju (bez odlaska u dubinu)...(ne bi to spadalo u rezultate?) mo�ete staviti u definiranju kreterija odlu�ivanja- \\ Koprivnica bez susjednih putovanja i koprivnica sa susjednim}
%\todo[inline]{$-$ evo ZA�TO JE TO BOLJE (bez i�ega drugog, kako su to radili drugi)}}
%\todo[inline]{ $-$ Komentirati dobivene rezultate sa stajali�ta apsolutne i relativne to�nosti matrica.}


\begin{comment}
paket trajectories\\
%iskoristiti completeness mjeru iz \cite{Chen:2019.}\\

- simulirati GPS putanje (50 komada) -jedan dan? jedan tjedan - 50 x 7 komada?\\
- simulirati prebrojavanje na n to�aka iz tih putanja\\
- simulirati CDR putanje iz tih putanja\\ 

- procijeniti matrice iz prebrojavanja\\
- procjeniti matrice iz CDR putanja\\

T = 15 min, T = 1h, T = 3 h \\
- histogrami po vremenskim okvirima, malo putovanja $\rightarrow$ previ�e zrnato!\\

- izbaciti putovanja u susjedne �elije ?\\

\cite{Chen:2019.}
\end{comment}

% !TeX encoding = windows-1250
\chapter{Rezultati}
%\todo[inline]{izno�enje rezultata i analiza rezultata do kojih je do�lo rje�avanjem problematike zadatka, rezultati mjerenja}
%\textit{ u poglavlju 6 �ete staviti slike i napisati rezultate zbrajanja- ne vi�e od toga.} 
\newpage
\begin{figure}[H]
	\begin{center}
		\includegraphics[width=17cm,keepaspectratio=true]{A_0_24_lmat}
		\caption{POM A}
		\label{fig:A}
	\end{center}
\end{figure}

\begin{figure}[H]
	\begin{center}
		\includegraphics[width=17cm,keepaspectratio=true]{B_0_24_lmat}
		\caption{POM B}
		\label{fig:B}
	\end{center}
\end{figure}


\begin{table}[!htpb]
	\renewcommand{\arraystretch}{1.2}
	\caption{Tablica usporedbe}
	\centering
	\hskip-2.0cm
	%\tiny
	\begin{tabular}{|c|c|c|c|c|c|c|c|c|}
		\hline
		POM
		&\makecell{Prostorno\\Obuhva�anje}
		&\makecell{Gusto�a\\Informacija}
		&\makecell{Prostorna\\Zrnatost}&\makecell{Vremenska\\Zrnatost}
		%&\makecell{Tematska\\Zrnatost}
		%&\makecell{Tematska\\Rezolucija\\Svrhe}
		&\makecell{Tematska\\Rezolucija\\Na�ina kretanja}
		&\makecell{Ukupna\\�irina\\Toka}
		\\ [0.5ex]
		\hline \hline
		A& \cellcolor{green!25}\makecell{neovisno o\\prometnoj\\infrastrukturi}
		&0.008830772
		&$1090\times1090$
		&8
		%&0
		%&0
		&0
		&30878
		\\ [0.5ex]
		\hline
		B
		&\makecell{ \\cesta\\  }
		&\cellcolor{green!25}0.08683223
		&$1090\times1090$
		&8
		%&0
		%&0
		&\cellcolor{green!25}1
		&\cellcolor{green!25}427646
		\\ [0.5ex]
		\hline
		
	\end{tabular}
\end{table}

% !TeX encoding = windows-1250
\chapter{Diskusija}


Podru�je vrednovanja Polazi�no-Odredi�nih Matrica, osobito onih iz suvremenih izvora, prili�no je neodre�eno. 
Ovaj rad uvodi novi pogled na usporedbu i vrednovanje \gls{pom} dobivenih iz razli�itih izvora i razli�itim postupcima procjene temeljen na vlastitim definicijama
parametara kvalitete. Osim parametara kvalitete, rad definira uvjete usporedbe dvije \gls{pom}-e te postupak odlu�ivanja o njihovoj odnosnoj kvaliteti. Rezultat istra�ivanja jest metodologija za usporedbu kontekstualiziranih \gls{pom}-a. Rezultati istra�ivanja podr�avaju po�etnu tezu.

Okvirna metodologija predstavljena i demonstrirana u ovom radu u postupku odlu�ivanja definira svaki parametar kvalitete kao jednako vrijedan, �to ima veliki utjecaj na kona�ne rezultate usporedbe. Druga�ije postavljeni odnosi parametara, primjerice dodjeljivanjem razli�ite \textit{te�ine} svakom parametru utjecali bi na kona�ni rezultat.

Procijenjeni nedostaci metodologije zahtijevaju
dublju analizu svakog definiranog parametra kvalitete i kriterija usporedbe zasebno, kao i njihovih me�usobnih ovisnost. 

Zadatak za budu�a istra�ivanja je i pronala�enje optimalnih vrijednosti parametara. 

Analiza matrica svedenih s apsolutnih na relativne vrijednosti.

\begin{equation}
X \rightarrow Z = \frac{x - \bar{x}}{\sigma}
\end{equation}

Odlomak \ref{trip_def} napominje da se putovanja dodjeljuju vremenskom okviru u kojem zapo�inju ili zavr�avaju bez obzira �to se prote�u kroz vi�e perioda. \textit{Klize�i prozor} zami�ljen je kao pomi�ni vremenski okvir koji obuhva�a samo putovanja koja mu u cijelosti pripadaju tj. zapo�inju i zavr�avaju unutar vremenskog okvira. Klize�i iz razloga �to razdoblje promatranja ne mora biti podijeljeno u odre�en broj vremenskih okvira ve� se vremenski okvir mo�e postaviti bilo kad. 
Analiza kako i koliko \textit{klize�i prozori} mogu utjecati na kvalitetu matrice. %Markovljevi lanci

\todo[inline]{
	
	Diskusija, zna�i uspostavu relevantnosti rezultata znanstvenog istra�ivanja u cjelini znanstvenih znanja znanstvene discipline, grane, polja i podru�ja znanosti, �ime je stvorena osnova za definiranje znanstvenog doprinisa znanstvenog rada, ali i znanstvenog istra�ivanja o kojemu rad izvje�tava. 
	
	Diskusija je poglavlje u kojemu se samo i isklju�ivo interpretiraju i komentiraju dobiveni rezultati.
	\\� Ovo je poglavlje u kojemu autori izla�u �to oni sami
	misle o zna�enju njihovih rezultata.
	\\� Podr�avaju li rezultati istra�ivanja njihovu po�etnu
	hipotezu ili ne i za�to? Tamo gdje je to neophodno
	potrebno je pozvati se na podatke, tablice, grafove i slike
	navedene u Rezultatima.
	\\� Autori diskutiraju mogu�e razloge zbog �ega su u svojim istra�ivanjima dobili navedene rezultate, u kojoj su mjeri
	kori�tene metode utjecale na rezultate te da li bi neke
	druge metode dale druk�ije rezultate. 
	\\Napokon, autori u diskusiji komentiraju kako se
	njihovi rezultati uklapaju u �iri kontekst
	znanstvenih znanja. Diskusija je pravo mjesto da
	se samokriti�ki upozori i na neke nedostatke
	(objektivne ili subjektivne) vlastite studije.\\
	� Zadatak autora je da u Diskusiji da
	najvjerojatnije, odnosno najbolje mogu�e
	obja�njenje s obzirom na rezultate koje je dobio
	(nitko ne o�ekuje otkrivanje apsolutnih istina i
	kona�nih rje�enja problema). 
	\\� Eksperimenti ili opa�anja ne�e uvijek potvrditi po�etnu
	hipotezu, ne�e potvrditi postojanje zna�ajne razlike
	izme�u eksperimentalnih rezultata i kontrolnih rezultata,
	ne�e se prona�i veza izme�u dviju varijabli ili postojanje
	trenda.
	\\� Ovakvi "negativni" rezultati su tako�er va�ni znanstveni
	rezultati i oni tako�er tra�e obja�njenje. Vrlo �esto ovakvi
	neo�ekivani rezultati mogu preusmjeriti istra�ivanja u
	drugom pravcu koji �e se pokazati va�nijim i zna�ajnijim.
	\\� Mnoga su velika otkri�a nastala nakon grje�aka ili nakon
	dobivanja neo�ekivanih ("negativnih") rezultata
}

\todo[inline]{
	
	\textit{U diskusiji recite da ste uveli jedan svoj vlastiti pogled temeljen na vlastitim definicijama zato �to je cijelo podru�je prili�no neodre�eno, da ste vi napravili svoj okvir, egzaktne definicije pojmova, i da ste radili po tim definicijama, da ste napravili tezu, da ste po toj tezi definirali elemente za usporedbu i postupak odlu�ivanja u kojem ste rekli, ako po ovom parametru dobijemo da je vrijednost ve�a da je matrica bolja i da ste na temelju toga \textbf{izveli odre�ene zaklju�ke}. Ka�ete �to vam je procijenjeni nedostatak: da niste obuhvatili ili potrebno je jo� istra�iti koliko klize�i prozori mogu utjecati na kvalitetu matrice, dali postoji neka mogu�nost pobolj�anja u samom postupku kreiranja marica (nemojte zalaziti puno u kreiranje i i�i pisati ono �to ste radili na kampu).}
	
	Pomi�ni vremenski okvir, analiza dijelova matrice prema parametrima ( dio koji je bolji, dio koji je slabiji) $->$ ispitati prisustvo neuniformnosti
	Neuniformnost  i rezolucija, nema "konstantnog uzorkovanja"

\url{http://www.unizd.hr/portals/4/nastavni_mat/1_godina/metodologija/PISANJE_ZNANSTVENOG_RADA.pdf}

}





\chapter{Zaklju�ak}
% Zaklju�ak � u kojemu se sa�imaju rezultati rada. U pravilu Zaklju�ak ne bi trebao biti du�i od dvije stranice. 



Establishing access policies for spatial information is a pressing societal need
requiring a better understanding of all values involved. It is complicated by the fact that information about indoor and geographic spaces gets collected and shared by almost everybody. This phenomenon of crowd-sourced or Volunteered Geographic Information (VGI, (Goodchild 2007)) is profoundly altering the values related to spatial information, from economic as well as institutional, ethical, and legal perspectives. 
http://www.economist.com/node/1788311
 % dati neko logicno ime umjesto ``Poglavlje_1''
%% !TeX encoding = windows-1250
\chapter{Metodologija usporedbe}

\section{Podaci}

\section{Rezultati}

 % itd.
%% !TeX encoding = windows-1250
\chapter{Zaklju�ak}
% Zaklju�ak � u kojemu se sa�imaju rezultati rada. U pravilu Zaklju�ak ne bi trebao biti du�i od dvije stranice. 


%\include{X_Uvod} 
%% !TeX encoding = windows-1250
\chapter{Postoje�e metrike kvalitete POM-e}
  
\section{Referentna matirca - \textit{Grand Truth matrix}}
goodness of fit measure
    
\section{Metrike}
    
\subsubsection{$R^{2}$ }
    (How close the models are to the reality, ...)
\subsubsection{GEH} 
	(How close the models are to the reality, ...)
    
\subsubsection{ MSE, SEM, RMSE, EBM (SAE)} 
    (How close the models are to the reality)
    
\subsubsection{Pearson korelacija redova}
	pearson correlation row to row 
    
    (Estimation of urban commuting paterns using cellphone network data)        

\subsubsection{Wasserstein Metric} (?)
    
\subsection{Strukturalna sli�nost}
    
\subsubsection{MSSI} 
    
    4.3.1.1.osnovni
    
    4.3.1.2.pobolj�ani 
%% !TeX encoding = windows-1250
\chapter{Odnosni parametri kvalitete}

%Parametar se definira tako i tako

Parametar - varijabla o kojoj ovisi odre�eni logi�ki izraz, matemati�ka formula ili funkcija, a koju promatramo kao dodatnu ovisnost u izrazu koji se definira kao da je ta vrijednost �vrsta.

\section{Zajedni�ki, objektivni kriteriji usporedbe}

-Isti grad\\
-Isto doba godine\\
-Isto vremensko razdoblje \\
-Week day, work day\\
-Ista ili sli�na definicija putovanja \\
-Isto pridodijeljena putovanja periodu \\
-Ima li matrica diagonalu? \\

(...)

%kada se preslikava bs �elije na drugi oblik, nastaju putovanja unutar novih �elija (koje su ve�e od bs �elija) \cite{Graells-Garrido:2016.}  

\section{Komparacijski indikatori}


\subsection{Definicija putovanja}

Kod generiranja matrica treba definirati ho�e li se putovanja koja se prote�u kroz vi�e perioda dodijeliti vremenskom okviru u kojem zapo�inju ili u kojem zavr�avaju. Iako ponekad nije specificirano, autori \cite{Bera:2011.} \cite{Filic:2016.} \cite{Gundlegard:2016.} sva putovanja dodjeljuju intervalu u kojem je putovanje zapo�eto. Teralytics u praksi generira obje vrste matrica \cite{Teralytics:2017.}.
\cite{Gundlegard:2016.} kako bi uvrstio �to vi�e putovanja, ne zanemaruje one zapise iz kojih nije jasno vidljiv po�etak putovanja ve� takva putovanja dodjeljuje na osnovu vjerojatnosti da su zapo�ela u odre�enom periodu. Sli�an postupak spominje se i u \cite{Toole:2015.} gdje za izra�un vjerojatnosti pose�u za nacionalnim anketama ili frekvencijom poziva.

Usporedba matrica iz razli�itih izvora (ankete i \gls{cdr}) s 2 razli�ite definicije putovanja (samo putovanja \gls{hw} i \gls{wh} te sva putovanja) pokazala se kao lo� pristup, razumljivo su matrice imale mali stupanj sli�nosti no neo�ekivano je i strukturalna sli�nost bila vrlo mala. \cite{Bonnel:2015.}



%The comparison between the matrices obtained from commuting and mobile phone data provide quite limited results. It is reasonable for the estimated number of trips to differ, as commuting data only relates to trips from the individual�s home to their place of work or study, while mobile phone data cover all trip purposes. But the analysis we have performed also show that the structures of the matrices have little in common. There is thus a high degree of dispersion in the rates of variation between the two sources of origin-destination data. This preliminary work was conducted on the basis of fairly strong hypotheses. It is therefore quite possible that more detailed analysis would make it possible to moderate some of the strongest hypotheses and improve comparability. Nevertheless, the only trip purposes covered by commuting data are work and study. It would therefore be necessary to be able to estimate the location of the home and place of work or study of the individuals from mobile phone data in order to significantly improve the matrices produced from this source. Even if a number of algorithms have been described in the literature (Chen et al., 2014; Calabrese et al., 2013; Phithakkitnukoon et al., 2012), our identification of these locations is bound to be uncertain unless we have a large number of events for each mobile phone, which is not the case with the data that we have used. It seems certain that the analysis of the mobile phone data from smartphones which are frequently connected to web-based applications would make it possible to attempt this type of analysis.


- prosje�an broj putovanja po stanovniku/ispitaniku Netko:nekad ne�to tipa 3 i 4.5, ugl ne�to ve�i broj za CDR \\


\begin{figure}[!htbp]
	\begin{center}
		\includegraphics[width=15cm,keepaspectratio=true]{short_n_long_journeys}
		\caption{** dodati caption **\cite{Gundlegard:2016.}}
		\label{fig:journeys}
	\end{center}
\end{figure}



\subsection{Prostorna razlu�ivost (Rezolucija)}

Prema hrvatskoj enciklopediji definicija razlu�ivosti (rezolucije) glasi: mjera za razaznavanje sitnih pojedinosti na nekom prikazu (npr. televizijskoj slici). U ra�unalstvu se odnosi na fino�u rasterske slike iskazanu ukupnim brojem slikovnih elemenata (relativna razlu�ivost) ili brojem slikovnih elemenata po in�u (stvarna razlu�ivost).
(...)\\
Kod \gls{pom} ...\\
Rezolucija �e ovisiti o to�nosti polo�aja! ...\\

Voronoi �elije tako�er je mogu�e ugnijezditi (�to se �esto radi kod usporedbe s matricama dobivenim drugim postupkom),a pritom se mo�e u matricu zabilje�iti i aproksimacija internog toka, odnosno broja putovanja unutar nove �elije (popuniti dijagonalu). Postotak eksternih putovanja koji se gubi u procesu agregacije Voronoi �elija analiziran je  za matrice iz regije Picardie u Francuskoj.
Agregacijom na razinu \textit{Urban Areas} (podru�ja oko gradova) 85\% svih putovanja postaje internim putovanjima, a na razini \textit{Urban Cores} (podru�ja oko ve�ih gradova) �ak 97\% po�etno zabilje�enih putovanja je interno, te ostaje samo 3\% eksternih putovanja. \cite{Bahoken:2013.}. Sli�nu situaciju opisuje i \cite{Graells-Garrido:2016.} (Vidi sliku \ref{fig:diagonal})

\begin{figure}[!htbp]
	\begin{center}
		\includegraphics[width=10cm,keepaspectratio=true]{diagonal_CDR}
		\caption{** dodati caption ** \cite{Graells-Garrido:2016.}}
		\label{fig:diagonal}
	\end{center}
\end{figure}

%Ruralna podru�ja rje�e zahtijevaju agregiranje �elija.


Na temelju usporedbi matrica kretanja u 4 razli�ita grada, dobivenih iz \gls{cdr} i iz anketa, gdje je svaki grad u anketnim matricama imao svoju prostornu podjelu odnosno rezoluciju, Toole u svom radu \cite{Toole:2015.} (o�ekivano) zaklju�uje da sa \textbf{smanjenjem rezolucije (visokim stupnjem agregacije Voronoi �elija) stupanj korelacije s matricama iz ankete raste}. 


\begin{figure}[!htbp]
	\begin{center}
		\includegraphics[width=10cm,keepaspectratio=true]{spaghetti_effect}
		\caption{\textit{Spaghetti-effect} - problem koji se mo�e javiti kod grafi�kog prikaza tokova matrice visoke prostorne i niske vremenske rezolucije. Na slici su prikazani su jednodnevni tokovi.  \cite{Bahoken:2013.}}
		\label{fig:spaghetti}
	\end{center}
\end{figure}

\subsection{Vremenska razli�ivost}

Neki autori predla�u da kraj odnosno po�etak jednog dana bude u 3:00 kako bi putovanja zapo�eta prethodnog dana, a koja prelaze u novi dan, u analizi spadala u raniji dan, odnosno kako dobar dio putovanja ne bi bio izostavljen samo zato jer se prote�e kroz 2 dana \cite{Schneider:2013.}\cite{Toole:2015.} . 

%Daily trips are estimated from filtered users by analyzing consecutive observations at different stay points during a given time window. They begin by defining an effective day as a period between 3 am one morning and 3 am on the next consecutive morning. This definition is used to minimize the number of trips that are prematurely ended due to the assumption that users start and end each day at hom

%\cite{Gundlegard:2016.} 5 min
%U praksi Teralytics generira matrice od 15 min, 1h i 1 dan

\subsubsection{Doba dana}

\subsubsection{Dani u tjednu i praznici}
%utorak- najvi�e prometa \cite{Travassoli:2016.}

Istra�ivanja modernih pristupa generiranju matrica ukazuju na smanjenu mobilnost za neradnih dana u tjednu, najvi�e u nedjelju.\cite{Calabrese:2011.} Bitno je napomenuti da dio istra�ivanja isti�e da je vikendima tako�er zabilje�en i manji broj telekomunikacijske aktivnosti. \cite{Bahoken:2013.}

% We obtain an average of 5 trips per day during the weekday, and 4.5 during the weekend. This number is reasonable when compared to the US National Household Travel Survey2 which evaluated this number to be between 4.18 during weekdays and 3.86 during weekends3 .

Osim o�ekivane bitne razlike u radnim i neradnim danima, uo�eno je da je petak, kao zadnji radni dan u tjednu, bitno druga�iji od ostalih radnih dana. \cite{Travassoli:2016.} \cite{Calabrese:2011.}  \cite{Scepanovic:2015.} Uo�ena je poja�ana mobilnost petkom na razini regije (Boston Metropolitan Area) \cite{Calabrese:2011.} i na razini cijele dr�ave (Obale Bjelokosti) s pove�anjem od 35\% u odnosu na nedjelju \cite{Scepanovic:2015.} na temelju matrica izra�enih iz \gls{cdr}, no zanimljivo u�ena je smanjena mobilnost vezana za javni prijevoz (izvor za generiranje matrica \textit{Pametne kartice za javni prijevoz}) na razini regije (Southeast Queensland, Australija). Osim prosje�nog broja dnevnih putovanja, petak ima i ne�to druga�iji dnevni uzorak po satima (rast prometa prema kraju radnog vremena zapo�inje ranije). Ponovno, petkom je tako�er uo�ena i najvi�a razina telekomunikacijske aktivnosti \cite{Bahoken:2013.} \cite{Bonnel:2015.} �to ukazuje na mogu�u pristranost kada su u pitanju matrice generirane iz \gls{cdr}.
Rezultati nacionalne ankete u Sjedinjenim Ameri�kim Dr�avama pokazuju da je prosje�an broj dnevnih putovanja po stanovniku 4.18 tijekom radnog dana te 3.86 tijekom neradnih dana u tjednu, �to jest u skladu sa zaklju�cima modernih metoda.


\begin{comment}
\begin{figure}[!htbp]
\begin{center}
\includegraphics[width=10cm,keepaspectratio=true]{weekday}
\caption{  \cite{Bahoken:2013.}}
\label{fig:weekday}
\end{center}
\end{figure}
\end{comment}

%While the census gives only a static information about origin-destination flows, the OD matrices derived from mobile phone data allows us to appreciate the differences in travel demand over time. Figure 5(a) shows the total daily travel demand for 3 different weeks in October 2009. A weekly pattern clearly appears in the travel demand, with the minimum over weekends (especially sundays) and a maximum over fridays. Moreover, Figure 5(a) shows a particular change in travel demand in the second monday (day number 9 in figure), corresponding to Columbus Day. For a better look at this pattern, we plot the hourly travel demand for Columbus Day compared to the other mondays (see Figure 5(b)). We clearly see a higher travel demand in the first 2 hours of the day, followed by lower demand from 4 to 9, and from 12 to 20, due to the holiday.  \cite{Calabrese:2011.}

\subsubsection{Sezonske razlike}

%grafika str 61.
\cite{Kuhlm:2015.}

%Compared with traditional census data, our methodology to detect OD matrices from mobile phone traces has several advantages: It can capture the weekday and weekend patterns as well as seasonal variations. It can capture work trips and non-work trips, which is essential for trip chaining and activity based modeling. Zaklju�ak 

\cite{Calabrese:2011.}

\subsection{�irina toka}

Ukupan broj odlazaka/dolazaka po vremenskom okviru za cijelu matricu.\\
- Obuhva�a pje�a�ki promet\\
- veli�ina uzorka (postotak stanovni�tva)\\
- k-anonimizacija\\
- dakako za pojedinu matricu ovisi o prostornoj i vremenskoj rezoluciji \\
- o tome �e ovisiti i Zero-cells (Nul-�elije)\\
- samo najaktivniji korisnici \cite{Schneider:2013.} \cite{Toole:2015.} (pristranost?)\\
za razliku od izvla�enja najaktivnijih korisnika 
-\cite{Calabrese:2011.} uzima samo slu�ajno odabranih 25\% korisnika (njih 1 milijun) da pojednostavi analizu. ...Dobro �ta, da mo�e pje�ke na prste ra�unati?!\\
- broj aktivnih korisnika unutar jednog tjedna jako varira iz dana u dan.  \cite{Bonnel:2015.}\\
-faktor skaliranja spominje se u najmanje 2 istra�ivanja, u jednom se radi o x 10

%\subsection{Geometrija prostorne podjele}
%(ne)uniformna podjela ...
%Interpolacija - ima li smisla 2 razli�ite podjele?!

\subsubsection{Infrastruktura}
Obuhva�a pje�a�ki promet
%\subsubsection{Sredstvo kretanja}

\subsection{Gusto�a informacija - kontekst}


\section{Me�uovisnost parametara}
- Ukoliko je rezlucija mala (velike �elije) nema potrebe za preciznim definiranjem kraja putovanja \\
- o definiciji putovanja ovisit �e i zero-cells ?
- \cite{Gundlegard:2016.} str. 9 definicija putovanja i prostorna i vremenska rezolucija, DEFINICIJA PUOVANJA JE KLJU�NA\\
-Za analizu Peak houre - intervali od 5 minuta! \cite{Gundlegard:2016.} str. 10\\

- 30 min vremenski okvir\cite{Schneider:2013.} str. 6\\


\begin{comment}
2 modela
PoV (Predicted vs Observed)

Robusnost matrice ->una�anje �uma

\cite{Zhao:2017.} kad spominje TAD
\end{comment}

 % itd.

\include{Literatura}  % ovo je ime Bibtex datoteke koju korisnik kreira

%:::::::::: ukljucenje popisa kratica u tekst ::::::::
% Blok linija koda ispod ovoga generira ukljucenje popisa kratica u tekstu. Za uporabu, vidjeti Upute.
% Nije obavezno. Ako se ne zeli koristiti, onda ovaj blok staviti u komentar pomocu znaka %
\printglossary[type=\acronymtype]
\pagestyle{plain}
\begin{glossary}{Longest string}
	\input{Kratice}
\end{glossary}


%:::::::::::: blok za definiranje Sazetka/Abstracta rada 
\begin{abstract}
	% !TeX encoding = windows-1250
\vspace{5pt}

%:::::::::::::::::::::::::::::::::::::::::::::::::::::
%:::::::::::: HRVATSKI :::::::::::::::::::::::::::::::
\noindent

\glsfirst{pom} %Izvori�no-Odredi�na ili 
%\textit{eng. \gls{odm}} ili \textit{Trip Table} 
(POM) je alat koji omogu�uje opis i sustavnu statisti�ku procjenu migracija stanovni�tva na nekom podru�ju u zadanom prostorno-vremenskom okviru. \gls{pom}-a slu�i za opis grupne mobilnosti i mjerenje socio-ekonomske aktivnosti u nekoj regiji, a naj�e��e se koristi u prometnoj znanosti za analizu i strate�ko planiranje prometnog optere�enja i prometne infrastrukture. Postoje brojni tradicionalni pristupi procjeni i vrednovanju \gls{pom}, a dolaskom suvremenih izvora podataka o kretanju (zapisi usluga baziranih na lokaciji) omogu�en je razvoj suvremenih pristupa. 
Postupci procjene koji uklju�uju kontekstualizaciju dnevnih migracija daju uvid u motive odnosno svrhu kretanja stanovni�tva. Postoje�e metode vrednovanja \gls{pom}-a definiraju vrijednost nove \gls{pom} usporedbom i razmatranjem sli�nosti s postoje�om \gls{pom} za isto podru�je.  Postoji potreba da se definira i kvantizira kvaliteta \gls{pom}-e kroz objektivne parametre. Ovaj rad predstavlja alternativni pristup vrednovanju \gls{pom}, definira objektivne parametre kvalitete, uvjete usporedbe i postupak odlu�ivanja o kvaliteti \gls{pom}. Definirana metodologija obuhva�a kontekstualizaciju kao dio kvalitete \gls{pom}. Metodologija je demonstrirana na usporedbi dvije \gls{pom} dobivene iz razli�itih izvora razli�itim postupcima procjene.
%i tako pro�iruje podru�je primjene \gls{pom} i analize kretanja stanovni�tva op�enito. 
%Kontekstualizacija omogu�uje izdvajanje visoko repetitivnih, a time i predvidivih kretanja, od nepredvidivih kretanja

%Ovo je tekst u kojem se opi�e sa�etak va�ega rada. Tekst treba imati duh rekapitulacije �to je prikazano u radu, nakon �ega slijedi 3-5 klju�nih rije�i (zamijenite dolje postavljene op�enite predlo�ke rije�i nekim suvislim vlastitim klju�nim rije�ima).
%:::::::::::::::::::::::::::::::::::::::::::::::::::::

\vspace{5pt}
%
\noindent \textbf{\textit{Klju�ne rije�i} --- Polazi�no-odredi�na matrica, Izvori�no-Odredi�na Matrica, parametri kvalitete, usporedba, vrednovanje} 

%:::::::::::: KRAJ HRVATSKOG DIJELA :::::::::::::::::::


%::::::::::::::::::::::::::::::::::::::::::::::::::::::
%:::::::::::: ENGLESKI ::::::::::::::::::::::::::::::::

%\vspace{-10pt}
\section*{Abstract}
\vspace{-10pt}
%This is a text where a brief summary of your work is outlined. The text should have a sense of recap of what was presented in the thesis, followed by 3-5 keywords (replace the general keyword templates below with some meaningful keywords of your own) .
%:::::::::::::::::::::::::::::::::::::::::::::::::::::::

\vspace{5pt}
%
\noindent \textbf{\textit{Keywords} ---Origin-Destination Matrix, Trip Table, quality parameters, comparation, evaluation}

%::::::::::::::::::::::::::::::::::::::::::::::::::::::
%:::::::::::: KRAJ ENGLESKOG DIJELA :::::::::::::::::::

	  % sazetak rada i kljucne rijeci na HR i EN
\end{abstract}


%:::::::::::: PRILOZI (neobavezno) ::::::::::::::::::::
% ispod \appendix zaglavlja pomocu \include dodati poglavlja s prilozima
% ukoliko nemate priloga, ovaj blok linija staviti u komentar
\appendix
% !TeX encoding = windows-1250
\chapter{Metrike za vrednovanje polazi�no-odredi�ne matrice}
\label{dodatak_metrics}

\subsection{Metrike za procjenu sli�nosti polazi�no-odredi�ne matrice s referentnom}

Za procjenu kvalitete \gls{pom}-a dobivenih isklju�ivo anketranjem u radu \cite{Cools:2010.} kori�tena je mjera srednja apsolutna postotna pogre�ka (engl. Mean Absolute Percentage Error, MAPE), te je prikazano da se zadovoljavaju�a razina kvalitete takvih \gls{pom}-a posti�e ako uzorak obuhva�a 50\% populacije. Autor Cools isti�e va�nost kori�tenja drugih izvora uz ankete za izradu \gls{pom}-a.

U radu \cite{Bera:2011.} navedene su statisti�ke mjere relativna pogre�ka (engl. Relative Error, RE), devijacija ukupne potra�nje (engl. Total Demand Deviation, TDD), srednja apsolutna pogre�ka (engl. Mean Absolute Error, MAE), korijen iz srednje kvadratne pogre�ke (engl. Root Mean Square Error, RMSE) te najve�a mogu�a relativna pogre�ka (engl. Maximum Possible Relative Error, MPRE) i razina prometne potra�nje (engl. Travel Demand Scale, TDS) koji procjenjuju kvalitetu neovisno o referentnoj matrici (no MPRE ne dopu�ta pogre�ke u prebrojavanju prometa, dok TDS ovisi o topologiji mre�e i odabiru ruta) \cite{Djukic:2013.}.

U \cite{Frias-Martinez:2012.} kori�ten je \textit{Pearsonov koeficijent korelacije} -  $r$ da bi se utvrdila \textbf{sli�nost svakog retka \gls{pom}-e} dobivene iz CDR \textbf{s retkom referentne} (ukupni izlazni tok iz svake polazi�ne �elije). Isti postupak kori�ten je za kontekstualizirane \gls{hw} i \gls{wh} \gls{pom}-e  dobivene iz \gls{cdr} u usporedbi s referentnim \gls{pom}-ma dobivenim anketiranjem. %**

U svome radu \cite{Travassoli:2016.} navodi se nekoliko mjera - {$R^{2}$, Geoffrey E. Havers statistika (GEH),  korijen iz postotak srednje kvadratne pogre�ke (engl. Root Mean Squared Error percentage, \%RMSE), uvodi novu mjeru  Eigenvalue-Based Measure (EBM) (temeljenu na svojstvenim vrijednostima matrica) i procjenjuje pouzdanost \gls{pom}-e dobivene iz sustava automatskog prikupljanja podataka u javnom prijevozu (autobus, vlak i trajekt). Spominje i Wasserstein metriku, mjeru koja ne uspore�uje samo vrijednosti parova istih �elija (\textit{elementwise}). 
	
Spearmanov koeficijent korelacije ranga kori�ten je u \cite{Graells-Garrido:2016.} za procjenu sli�nosti \gls{pom}-a dobivenih iz \gls{cdr} sa tada aktualnim \gls{pom}-ma dobivenim anketiranjem.
	
	%(...)
	%Dynamic Travel Demand \cite{Gundlegard:2016.}
	
	\begin{comment}
	
	mozda po jednu recenicu za svako
	\subsubsection{$R^{2}$ }
	The R-squared (R2), as one of the most commonly and widely used (Washington et al., 2011), is a statistical measure of how close the data are to the fitted regression line, and itused for comparing between origin-destination pairs of two ODMs. R2 values rang from 0 to 1, with higher values indicating less difference between ODMs. 
	Along with considering higher value of R2 as a higher level of similarity, the regression line should be close to a 45-degree line through the origin. In this condition, the coefficient of the line should be closer to one and the intercept should be closer to zero. The lower and greater coefficient values indicate the tendency of the pattern to overestimate or underestimate values in the reference OD matrix.\cite{Travassoli:2016.}
	
	\subsubsection{GEH}
	Geoffrey E. Havers (GEH) statistic
	The GEH statistic is used to evaluate the level of closeness between origin-destination pairs of two ODMs. The GEH is applied to every pair in the two ODMs, with a GEH of less than 5 indicating a good fit (Hollander and Liu 2008). Then, the percentage of OD pairs that have a GEH equal to or less than 5 is calculated to indicate the level of closeness between two ODMs.
	\cite{Travassoli:2016.}   
	
	\subsection{RMSE i \%RMSE} 
	The root mean squared error (RMSE) and accordingly the percent root mean squared error(%RMSE) are used to evaluate the closeness of the ODMs. The %RMSE is where the variability of the demand is most evident: if two demand ODMs were identical, the %RMSE would be equal to zero.
	\subsubsection{ MSE, SEM, EBM (SAE)} 
	(How close the models are to the reality)
	
	\subsubsection{Pearson korelacija redova}
	
	\subsubsection{Sperman rank korelacija}
	
	\subsubsection{Wasserstein Metric} (?)
	
	\end{comment}
  % dati neko suvislo ime umjesto ovoga
% !TeX encoding = windows-1250
\chapter{Prilagodba prostorne podjele}
\label{dodatak_ppp}

Definiramo li to�nost matrice isklju�ivo na temelju sli�nosti svakog njenog elementa s ekvivalentnim elementom u \textit{grand truth} matrici nu�no je osigurati da prostorna podjela matrice odgovara prostornoj podjeli referentne ili je potrebno obje matrice svesti na zajedni�ku prostornu podjelu. 

\section{Objedinjenje baznih stanica}
Pregledom literature utvr�eno je da se koriste \textit{k-sredina} i \textit{mean-shift} algoritam za grupiranje baznih stanica u broj grupa jednak broju \gls{taz} na podru�ju interesa \cite{Goulding:2016.} \cite{Alhazzani:2016.} \cite{Mellegard:2011.} ili pak jedinicama samouprave (eng. \textit{municipalities}) kako bi  \gls{pom}-e u kona�nici bile usporedive s postoje�ima, dobivenim iz anketa \cite{Coscia:2015.} \cite{Frias-Martinez:2012.}. Nije utvr�ivano kolika se gre�ka uvodi ovim postupcima. Primjer gusto�e tornjeva baznih stanica na podru�jima anketnih jedinica prikazan je na slici \ref{fig:density}
%\cite{Mellegard:2011.} ne definira unaprijed broj grupa ve� za grupaciju koristi \textit{mean-shift} algoritam koji sam odre�uje broj grupa na podru�ju �vedske, potom koriste�i \gls{osm} grupe ve�e uz imena gradova i mjesta. 
%U istra�ivanju \cite{Coscia:2015.} bazne stanice na podru�ju Kolumbije grupiraju po jedinicama samouprave (eng. \textit{municipalities}). Isti postupak je kori�ten i u istra�ivanju na podru�ju Madrida, kako bi  \gls{pom}-e u kona�nici bile usporedive s onima dobivenim iz anketa \cite{Frias-Martinez:2012.}.

\begin{figure}[!htbp]
	
	\begin{center}
		\includegraphics[width=15cm,keepaspectratio=true]{Santiago_antenna_density}
		\caption{Prostorna podjela na anketne jedinice u gradu Santiagu. Gusto�a tornjeva baznih stanica po anketnim jedinicama. Gusto�a varira od 1 do 250 tornjeva po anketnoj jedinici.
			\cite{Graells-Garrido:2016.}}
		\label{fig:density}
	\end{center}
\end{figure}


\section{Interpolacija}
\label{interpol}
Postupak konverzije \gls{pom}-a iz prostorne podjele na Voronoi �elije u drugu prostornu podjelu, npr. uniformnu kvadratnu mre�u (�elije $1 km^2$), opisan u priru�niku \cite{Goulding:2016.}. U postupku odre�ivanja postotka ukupnog toka Voronoi �elije koji �e se dodijeliti novoj kvadratnoj �eliji predla�e uzeti u obzir: povr�inu preklapanja tih �elija, broj zgrada ili ukupnu povr�inu zgrada (uklju�uju�i katove) na podru�ju preklapanja tih �elija. 
Na slici \ref{fig:interpolation} je prikazan tre�i oblik interpolacije gdje su kori�teni podaci o ukupnoj kvadratnoj povr�ini zgrada. 

\begin{figure}[!htbp]
	\begin{center}
		\includegraphics[width=6.5cm,keepaspectratio=true]{interpolation}
		\caption{Primjer mozaika \textit{krhotina} za interpolaciju izme�u Voronoi �elija i kvadratne mre�e. Nijansa odra�ava postotak povr�ine zgrada u \textit{krhotini} u odnosu na cijelu Voronoi �eliju kojoj pripada koji �ini \textit{te�inski faktor} svake polazi�ne i odredi�ne \textit{krhotine} te pripadaju�ih ("roditeljskih") Voronoi �elija u izra�unu toka koji pripada krhotini. Sumom svih tokova krhotina jedne kvadratne zone izra�una se kona�ni tok te kvadratne zone. \cite{Goulding:2016.}}
		\label{fig:interpolation}
	\end{center}
\end{figure}

\section{Posljedice objedinjenja prostornih zona}

Voronoi �elije tako�er je mogu�e objediniti u ve�e zone. Pritom se mo�e u matricu bilje�iti i aproksimacija internog toka, odnosno broja putovanja unutar nove �elije (dijagonala matrice). Takva putovanja nazivaju se internim putovanjima. Studija \cite{Toole:2015.} je pokazala da sa \textbf{smanjenjem rezolucije (visokim stupnjem objedinjenja Voronoi �elija) stupanj korelacije s \gls{pom}-ma iz ankete (za isti grad) raste} (\gls{pom}-e dobivene iz \gls{cdr} i iz anketa u 4 razli�ita grada). 
Postotak eksternih putovanja koji se gubi u procesu objedinjenja Voronoi �elija analiziran je u studijama \cite{Graells-Garrido:2016.} (Vidi sliku \ref{fig:diagonal}) i \cite{Bahoken:2013.} za \gls{pom}-e iz regije Picardie u Francuskoj.
\textbf{Objedinjenjem Voronoi �elija na razinu \textit{Urban Areas} (podru�ja oko gradova) 85\% svih putovanja postaje internim putovanjima, a na razini \textit{Urban Cores} (podru�ja oko ve�ih gradova) �ak 97\% po�etno zabilje�enih putovanja je interno, te ostaje samo 3\% eksternih putovanja} \cite{Bahoken:2013.}.

\begin{figure}[!htpb]
	\begin{center}
		\subfloat[Izvor podataka anketa ]{\label{fig:survey}
			\includegraphics[height=9cm,keepaspectratio=true]{survey}}\\ 
		%\hspace{10pt}
		\subfloat[Izvor podataka \gls{cdr}]
		{\label{fig:cdr}
			\includegraphics[height=9cm,keepaspectratio=true]{cdr}}
		\caption{Distribucija \acrshort{cdr} putovanja na podru�ju Santiaga, �ile. POM je normalizirana prema redovima (polazi�tima) \cite{Graells-Garrido:2016.}}
		\label{fig:diagonal}
	\end{center}
\end{figure}
%Ruralna podru�ja rje�e zahtijevaju agregiranje �elija.


\chapter{OpenStreetMap}
\label{dodatak_osm}
\gls{osm} je svjetski ra�iren projekt koji kreira i pru�a slobodne zemljopisne podatke (zemljovide gradova i naselja) temeljen na volonterskom doprinosu zajednice. Projekt pru�a detaljne, a�urne digitalne zemljovide kompatibilne s \gls{gis} aplikacijama. 
Zapo�et je prije 15 godina u Ujedinjenom Kraljevstvu kao odgovor na tehni�ka ili pravna ograni�enja postoje�ih "slobodnih" baza prostornih podataka kao �to je GoogleMaps. Kartu svijeta na visokoj razini kvalitete i to�nosti odr�avaju zajedni�ki doprinos i me�usobna kontrola unosa.

Struktura osnovnih elemenata na toj rasterskoj karti je hijerarhijska. Element u hijerarhiji mo�e biti: �vor (engl. \textit{node}), put (engl. \textit{way}) ili relacija (engl. \textit{relation}). Uz hijerarhijske postoji i opisni element koji se naziva oznakom (engl. \textit{tag}), a njegova funkcija je opisati zna�ajke hijerarhijskog elementa uz koji je vezan. 

�vor je jedinstvena to�ka u prostoru (sa identifikacijskom oznakom, zemljopisnom �irinom i du�inom) koja naj�e��e predstavlja fizi�ki objekt (zgrada, dio ceste...) te sadr�i jedan ili vi�e \textit{klju� = vrijednost} oznaka (engl. \textit{key = value tag}) koji definiraju razne zna�ajke objekta.

Put je ure�ena lista 2 do 2,000 �vorova, koja tako�er mo�e biti opisana \textit{klju� = vrijednost} oznakama. 

Relacija je ure�ena lista �vorova, puteva i/ili relacija. Definira logi�ku ili geografsku povezanost �lanova.
 
Primjeri oznaka vezanih za �vorove:  \textit{office=company}, \textit{building=residential}, \textit{building=hotel}, \textit{building=church}, \textit{leisure=sports\_centre}, \textit{amenity=school}. Primjer �iga vezanog uz put \textit{highway=residential}.

\gls{osm} doprinosi istra�ivanjima analize kretanja stanovni�tva i kao izvor prometne infrastrukture, njenih meta podataka (kolnik, mostovi, ograni�enja brzine i dr.)\cite{Toole:2015.} i podataka o podnoj povr�ini objekata (engl. \textit{floor area}) \cite{Goulding:2016.}. 
%Raspodjela toka po prometnoj infrastrukturi nadilazi podru�je ovog rada no jest dio problematike modeliranja prometa.  
\chapter{Procjena Polazi�no-Odredi�ne Matrice B}
\label{dodatak_taxi}
Podatci u formatu \ref{fig:TaxiData} su grupirani prema identifikacijskoj oznaci (Taxi Id) i sortirani prema vremenskom �igu (Time). Po�etak putovanja definiran je kao prelazak statusa ( Occupancy Status) iz $0$ u $1$, a kraj kao prelazak iz 1 u 0. Prostorna podjela POM je na Voronoi �elije oko tornjeva baznih stanica jer podatci o \gls{taz} nisu javno dostupni. Generirano je 8 \gls{pom}, a sva putovanja su dodjeljivana periodu u kojem po�inju.

\begin{figure}[!htbp]
	\begin{center}
		\includegraphics[width=10cm,keepaspectratio=true]{TaxiData}
		\caption{Format podataka iz kojih je procijenjena POM B}
		\label{fig:TaxiData}
	\end{center}
\end{figure}




\end{document}
