% !TeX encoding = windows-1250
\chapter{Odnosni parametri kvalitete}

%Parametar se definira tako i tako

Parametar - varijabla o kojoj ovisi odre�eni logi�ki izraz, matemati�ka formula ili funkcija, a koju promatramo kao dodatnu ovisnost u izrazu koji se definira kao da je ta vrijednost �vrsta.

\section{Zajedni�ki, objektivni kriteriji usporedbe}

Kod generiranja matrica treba definirati ho�e li se putovanja koja se prote�u kroz vi�e perioda dodijeliti vremenskom periodu u kojem zapo�inju ili u kojem zavr�avaju. Iako ponekad nije specificirano, \cite{Bera:2011.} \cite{Filic:2016.} sva putovanja dodjeljuju intervalu u kojem je putovanje zapo�eto.

Isti grad\\
Isto doba godine\\
Isto vremensko razdoblje

Week day, work day

-Ima li matrica diagonalu? - kada se preslikava bs �elije na drugi oblik, 
nastaju putovanja unutar novih �elija (koje su ve�e od bs �elija) \cite{Graells-Garrido:2016.}  

\section{Komparacijski indikatori}
\subsection{Vremenski okvir}
Departure/Arrival time
\subsection{Razlu�ivost (Rezolucija)}

Prema hrvatskoj enciklopediji definicija razlu�ivosti (rezolucije) glasi: mjera za razaznavanje sitnih pojedinosti na nekom prikazu (npr. televizijskoj slici). U ra�unalstvu se odnosi na fino�u rasterske slike iskazanu ukupnim brojem slikovnih elemenata (relativna razlu�ivost) ili brojem slikovnih elemenata po in�u (stvarna razlu�ivost).
to�nost polo�aja

\subsection{�irina toka}
Ukupan broj odlazaka/dolazaka po vremenskom okviru za cijelu matricu

\subsection{Geometrija prostorne podjele}
(ne)uniformna podjela
\subsection{Definicija putovanja}

\subsection{}
\subsubsection{Infrastruktura}
\subsubsection{Sredstvo kretanja}

\subsection{Gusto�a informacija - kontekst}


\section{Me�uovisnost parametara}
Ukoliko je rezlucija mala (velike �elije) nema potrebe za preciznim definiranjem kraja
\begin{comment}
2 modela
PoV (Predicted vs Observed)

Robusnost matrice ->una�anje �uma

\end{comment}