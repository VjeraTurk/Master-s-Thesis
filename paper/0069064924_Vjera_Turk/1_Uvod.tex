% !TeX encoding = windows-1250
\chapter{Uvod}

Polazi�no-odredi�na matrica (POM) omogu�uje sustavnu statisti�ku procjenu migracija stanovni�tva u zadanom prostorno-vremenskom okviru. Za razliku od tradicionalnog pristupa brojanja putovanja i putnika, za procjenu POM-e danas se sve vi�e koristi statisti�ka analiza podataka iz suvremenih informacijskih i komunikacijskih sustava (zapisi o aktivnostima u javnoj pokretnoj mre�i, zdru�ena o�itanja prijamnika za satelitsku navigaciju i sl.), �ime je omogu�eno pobolj�anje kvalitete procjene preslikavanjem POM-e na kontekst. 

Pojavljuje se potreba za objektivnom procjenom kvalitete POM-e u odnosu na referentnu (kontrolnu). U ovom radu definirani su odnosni parametri kvalitete POM-e te je razvijena metodologija usporedbe dviju POM-a dobivenih razli�itim postupcima procjene i s podatcima iz razli�itih izvora. Usporedba je obavljena kori�tenjem numeri�kog i grafi�kog oblika POM-e. 

Metodologija je izvedena u programskom okru�enju za statisti�ko ra�unarstvo R te je demonstrirana njena primjena na slu�aju usporedbe dviju POM-a. Dobiveni rezultati komentirani su sa stajali�ta apsolutne i relativne to�nosti matrica.



\chapter{Polazi�no-odredi�na matrica}
Tranzitna, t-POM
koncentracija u radu na CDR (?)
            
\section{Tradicionalni pristupi generiranju POM-a}

\subsection{Ankete}
cijena anketiranja  (u jednom od radova 10 eura po ispitaniku?)
\subsection{Prebrojavanje vozila}
ru�no, video
\subsection{Gravitacijski model}
\subsection{Problematika i ograni�enja tradicionalnih na�ina}
Zastarijevanje 

\section{POM iz zapisa o aktivnostima u javnoj pokretnoj mre�i}

\subsection{Razlike u pristupima}
        Tranzitna t-POM,

		Definiranje putovanja
	        Period - Departure/Arrival time
        
        Grad, Dr�ava
    
        CDR- POM u zemljama u razvoju

\subsection{To�nost polo�aja}
 to�nost polo�aja - aproksimacija s BS ili signal strenght (RSSI)
 + neka ona drugo baza LTO(?)

\subsection{Geometrija prostorne podjele}

\subsubsection{Heksagoni}
\subsubsection{Voronoi tesalacije}
\subsubsection{Jedinice samouprave}
\subsubsection{Pravokutna mre�a}

\subsection {Dobre prakse u generiranju POM iz CDR}
Skaliranje CDR POM (primjerak $->$ pouplacija)
(linking to transport infrastructure?)
k-anonymization

\newpage
\section{Drugi primjeri automatskog prikupljanja}

\subsection{GNSS}
\subsection{Javni prijevoz}

\vspace{10pt}