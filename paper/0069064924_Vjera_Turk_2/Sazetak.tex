% !TeX encoding = windows-1250
\vspace{5pt}

%:::::::::::::::::::::::::::::::::::::::::::::::::::::
%:::::::::::: HRVATSKI :::::::::::::::::::::::::::::::
\noindent

\glsfirst{pom} %Izvori�no-Odredi�na ili 
%\textit{eng. \gls{odm}} ili \textit{Trip Table} 
(POM) je alat koji omogu�uje opis i sustavnu statisti�ku procjenu migracija stanovni�tva na nekom podru�ju u zadanom prostorno-vremenskom okviru. \gls{pom}-a slu�i za opis grupne mobilnosti i mjerenje socio-ekonomske aktivnosti u nekoj regiji, a naj�e��e se koristi u prometnoj znanosti za analizu i strate�ko planiranje prometnog optere�enja i prometne infrastrukture. Postoje brojni tradicionalni pristupi procjeni i vrednovanju \gls{pom}, a dolaskom suvremenih izvora podataka o kretanju (zapisi usluga baziranih na lokaciji) omogu�en je razvoj suvremenih pristupa. 
Postupci procjene koji uklju�uju kontekstualizaciju dnevnih migracija daju uvid u motive odnosno svrhu kretanja stanovni�tva. Postoje�e metode vrednovanja \gls{pom}-a definiraju vrijednost nove \gls{pom} usporedbom i razmatranjem sli�nosti s postoje�om \gls{pom} za isto podru�je.  Postoji potreba da se definira i kvantizira kvaliteta \gls{pom}-e kroz objektivne parametre. Ovaj rad predstavlja alternativni pristup vrednovanju \gls{pom}, definira objektivne parametre kvalitete, uvjete usporedbe i postupak odlu�ivanja o kvaliteti \gls{pom}. Definirana metodologija obuhva�a kontekstualizaciju kao dio kvalitete \gls{pom}. Metodologija je demonstrirana na usporedbi dvije \gls{pom} dobivene iz razli�itih izvora razli�itim postupcima procjene.
%i tako pro�iruje podru�je primjene \gls{pom} i analize kretanja stanovni�tva op�enito. 
%Kontekstualizacija omogu�uje izdvajanje visoko repetitivnih, a time i predvidivih kretanja, od nepredvidivih kretanja

%Ovo je tekst u kojem se opi�e sa�etak va�ega rada. Tekst treba imati duh rekapitulacije �to je prikazano u radu, nakon �ega slijedi 3-5 klju�nih rije�i (zamijenite dolje postavljene op�enite predlo�ke rije�i nekim suvislim vlastitim klju�nim rije�ima).
%:::::::::::::::::::::::::::::::::::::::::::::::::::::

\vspace{5pt}
%
\noindent \textbf{\textit{Klju�ne rije�i} --- Polazi�no-odredi�na matrica, Izvori�no-Odredi�na Matrica, parametri kvalitete, usporedba, vrednovanje} 

%:::::::::::: KRAJ HRVATSKOG DIJELA :::::::::::::::::::


%::::::::::::::::::::::::::::::::::::::::::::::::::::::
%:::::::::::: ENGLESKI ::::::::::::::::::::::::::::::::

%\vspace{-10pt}
\section*{Abstract}
\vspace{-10pt}
%This is a text where a brief summary of your work is outlined. The text should have a sense of recap of what was presented in the thesis, followed by 3-5 keywords (replace the general keyword templates below with some meaningful keywords of your own) .
%:::::::::::::::::::::::::::::::::::::::::::::::::::::::

\vspace{5pt}
%
\noindent \textbf{\textit{Keywords} ---Origin-Destination Matrix, Trip Table, quality parameters, comparation, evaluation}

%::::::::::::::::::::::::::::::::::::::::::::::::::::::
%:::::::::::: KRAJ ENGLESKOG DIJELA :::::::::::::::::::

