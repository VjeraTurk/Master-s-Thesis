% !TeX encoding = windows-1250
\vspace{5pt}

%:::::::::::::::::::::::::::::::::::::::::::::::::::::
%:::::::::::: HRVATSKI :::::::::::::::::::::::::::::::
\noindent

\glsfirst{pom} %Izvori�no-Odredi�na ili 
%\textit{eng. \gls{odm}} ili \textit{Trip Table} 
(POM) je alat koji omogu�uje opis i sustavnu statisti�ku procjenu migracija stanovni�tva na nekom podru�ju u zadanom prostorno-vremenskom okviru. \gls{pom}-a slu�i za opis grupne mobilnosti i mjerenje socio-ekonomske aktivnosti u nekoj regiji, a naj�e��e se koristi u prometnoj znanosti za analizu i strate�ko planiranje prometnog optere�enja i prometne infrastrukture. Postoje brojni tradicionalni pristupi procjeni i vrednovanju \gls{pom}-a, a dolaskom pasivno prikupljanih suvremenih izvora podataka o kretanju (zapisi usluga baziranih na lokaciji) omogu�en je razvoj suvremenih pristupa i pro�irenje podru�ja primjene. 
Postupci procjene koji uklju�uju kontekstualizaciju dnevnih migracija daju uvid u motive odnosno svrhu kretanja stanovni�tva. Postoje�e metode vrednovanja \gls{pom}-a definiraju vrijednost nove \gls{pom} usporedbom i razmatranjem sli�nosti s postoje�om za isto podru�je.  Postoji potreba da se definira i kvantizira kvaliteta \gls{pom}-e kroz objektivne parametre. Ovaj rad predstavlja alternativni pristup vrednovanju \gls{pom}-a, definira objektivne parametre kvalitete, uvjete usporedbe i postupak odlu�ivanja o kvaliteti \gls{pom}. Definirana metodologija obuhva�a kontekstualizaciju kao dio kvalitete \gls{pom}. Metodologija je demonstrirana na usporedbi dvije \gls{pom} dobivene iz razli�itih izvora razli�itim postupcima procjene.
%i tako pro�iruje podru�je primjene \gls{pom} i analize kretanja stanovni�tva op�enito. 
%Kontekstualizacija omogu�uje izdvajanje visoko repetitivnih, a time i predvidivih kretanja, od nepredvidivih kretanja

%Ovo je tekst u kojem se opi�e sa�etak va�ega rada. Tekst treba imati duh rekapitulacije �to je prikazano u radu, nakon �ega slijedi 3-5 klju�nih rije�i (zamijenite dolje postavljene op�enite predlo�ke rije�i nekim suvislim vlastitim klju�nim rije�ima).
%:::::::::::::::::::::::::::::::::::::::::::::::::::::

\vspace{5pt}
%
\noindent \textbf{\textit{Klju�ne rije�i} --- Polazi�no-Odredi�na Matrica (Izvori�no-Odredi�na Matrica), kontekstualizacija, parametri kvalitete, usporedba, vrednovanje} 

%:::::::::::: KRAJ HRVATSKOG DIJELA :::::::::::::::::::


%::::::::::::::::::::::::::::::::::::::::::::::::::::::
%:::::::::::: ENGLESKI ::::::::::::::::::::::::::::::::

%\vspace{-10pt}
\section*{Abstract}
\vspace{-10pt}
%This is a text where a brief summary of your work is outlined. The text should have a sense of recap of what was presented in the thesis, followed by 3-5 keywords (replace the general keyword templates below with some meaningful keywords of your own) .
%:::::::::::::::::::::::::::::::::::::::::::::::::::::::
Origin-Destination Matrix (ODM) is a tool that enables description and systematic statistical estimation of human migrations in an area and within a give space-time frame. ODM is used to describe group mobility and measure socio-economic activity in a region, and is most commonly used in transport science for the analysis and strategic planning of transport load and transport infrastructure. There are a number of traditional approaches to ODM estimation and evaluation.  The rise of passively collected location data (from location based services) has enabeld development of new, modern approaches and broadens the field of use. Approaches that involve contextualisation of daily migrations provide insight into the motives and te purpose of urban movements. The existing ODM valuation methods define the value of a new ODM by comparing it to an existing ODM and conidering the simmilarity. There is a need to define and quantify the quality of ODM through objective parameters. This thesis presents an alternative approach to ODM evaluation, defines objective quality parameters, comparison conditions and decision process. Defined methodology captures contextualization as part of ODM quality. The methodololgy is demonstrated with comparing two ODMs obtained from different sources and with different estimation procedures.

\vspace{5pt}
%
\noindent \textbf{\textit{Keywords} ---Origin-Destination Matrix (Trip Table), contextualization, quality parameters, comparation, evaluation}

%::::::::::::::::::::::::::::::::::::::::::::::::::::::
%:::::::::::: KRAJ ENGLESKOG DIJELA :::::::::::::::::::

