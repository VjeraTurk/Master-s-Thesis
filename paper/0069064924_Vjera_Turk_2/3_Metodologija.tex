% !TeX encoding = windows-1250
\chapter{Metodologija}
\todo[inline]{opis sklopa/programske podr�ke. Opis rje�avanja zadatka i odgovaraju�i prora�uni}

%Studying the behaviour of moving objects over time and their interaction, either between objects or with environment, plays a crucial role in understanding how they use space and more importantly how they interact each other. Moving objects are moving within a particular area over time, thus a snapshot of a trajectory pattern might be seen as a spatial point pattern. This aspect then empowers us to study the behaviour of moving objects within space and over time. A set of locations, usually non-uniformly distributed within a certain region, can be considered as a realisation of a spatial point process.

\section{Uvjeti usporedbe}
\gls{pom}-e koje se uspore�uju trebaju zadovoljavati uvjete usporedbe. Rije� je o slijede�im zajedni�kim obilje�jima:
\begin{itemize}
	\item  isto podru�je/grad/dr�ava
	\item  isto doba godine
	\item vremensko razdoblje iste duljine
	\item  ako se radi o jednom danu - idealno isti dan u tjednu
	\item ista kategorija putovanja (tranzitna, \textit{komutacijska}, kontekstualizirana)
	\item obje \gls{pom}-e napravljene objektivno na cijeloj populaciji (reprezentativnom uzorku)
	
\end{itemize}

\section{Brzina kretanja}
\todo[inline]{
- Analiza prikladnosti odabira perioda i dimenzije �elija s obzirom na brzinu kretanja agenata ?!
\\- lo�e postavke one s kojima �e putovanja nu�no zavr�iti isklju�ivo u istim/ susjednim �elijama?!
- avemove() funkcija iz trajectories?
}

\begin{comment}
paket trajectories\\
%iskoristiti completeness mjeru iz \cite{Chen:2019.}\\

- simulirati GPS putanje (50 komada) -jedan dan? jedan tjedan - 50 x 7 komada?\\
- simulirati prebrojavanje na n to�aka iz tih putanja\\
- simulirati CDR putanje iz tih putanja\\ 

- procijeniti matrice iz prebrojavanja\\
- procjeniti matrice iz CDR putanja\\

T = 15 min, T = 1h, T = 3 h \\
- histogrami po vremenskim okvirima, malo putovanja $\rightarrow$ previ�e zrnato!\\

- izbaciti putovanja u susjedne �elije ?\\

\cite{Chen:2019.}
\end{comment}
