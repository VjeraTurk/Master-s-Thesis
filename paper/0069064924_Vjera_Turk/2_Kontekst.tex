% !TeX encoding = windows-1250
\chapter{Kontekstualizirane matrice}

\section{Kontekst iz samog izvora podataka o kretanju}   
	Kontekst izvu�en iz CDR (long term modeli- CDR vi�e mjeseci/tjedana)
   HBW,HBO,NHB (The path most Traveled, HBW,WBH,HBO,NBO (Best Practices), 
   HWO (?), HWHA (?) 
   HW WH (Estimating origin-Destination flows using opportunistically collected 
   mobile phone location data from one milion users in Boston Metropolitan Area)
   (2012 Estimation of urban commuting patterns using cellphone network data)
   - "MODE" kao kontekst, car/public transport/flight Terralytics

\section{Kontekst iz vanjskih izvora}

Some studies combined human mobility with land use or POIs data to segment districts in urban areas according to their functions or use. The type of data used to capture human mobility behavior varies between individual GPS traces [10, 11], taxi pick up/drop off locations as in [7, 12] , Call Detail Records (CDRs) as in [2, 8], social media check ins as
in [13�15], and bus smart card data as in [16]. 

\subsection{\textit{Points of Interes}} 
\gls{pois} Points of Interes (Urban Attractors ima 22 kategorije koristi bazu s POI)
\subsection{Open Street Map}
  \subsubsection{Model raspodjele toka (na� model)}
   \subsubsection{Izvor infrastrukture} OSM kao izvor infrastrukture 
   (jedan od radova koristi za broj traka na auto cesti jedan za provjeru naseljenosti podru�ja/broj katova zgrada)

\section{Sredstvo (na�in) kretanja}
   