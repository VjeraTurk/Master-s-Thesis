% !TeX encoding = windows-1250
%%%%%%%%%%%%%%%%%%%%%%%%%%%%%%%%%%%%%%%%%%%%%%%%%%%%%%%%%%%%%%%%%%%%%%%%%%%%%%%%%%%%%%%%%%%%%%%%%
%% ovo je jednostavniji (ali neautomatski) primjer definiranja liste akronima, a student neka to zamijeni svojim kraticama i doda sve koje zeli
%% ako se ne zeli deklarirati popis kratica, onda ga staviti pod komentar u glavnoj datoteci
%
%   \item[{\bf HTML}]	Hypertext Markup Language
%   \item[{\bf AJAX}]	Asynchronous JavaScript and XML

%%%%%%%%%%%%%%%%%%%%%%%%%%%%%%%%%%%%%%%%%%%%%%%%%%%%%%%%%%%%%%%%%%%%%%%%%%%%%%%%%%%%%%%%%%

%%%%%%%%%%%%%%%%%%%%%%%%%%%%%%%%%%%%%%%%%%%%%%%%%%%%%%%%%%%%%%%%%%%%%%%%%%%%%%%%%%%%%%%%%%
% Sofisticiraniji nacin definiranja i uporabe kratica je preko sljede�e sintakse

 \addcontentsline{toc}{chapter}{Pojmovnik}
% primjeri definicije kratica
% ove pojmove zamijenite nekim svojima i po tom predlosku nadogradite listu po potrebi
\newacronym{pom}{POM}{Polazi�no-Odredi�na Matrica (Izvori�no-odredi�na matrica)}
\newacronym{iom}{IOM}{Izvori�no-Odredi�na Matrica}
\newacronym{odm}{ODM}{Origin-Destination Martix} 
\newacronym{cdr}{CDR}{Call Data Records ili Charging Data Records} 
\newacronym{gsm}{GSM}{Global System for Mobile (Communications)}
\newacronym{rsi}{RSI}{Road Side Interview}
\newacronym{pois}{POIs}{Points of Interest}
\newacronym{hw}{HW}{Home-Work}
\newacronym{wh}{WH}{Work-Home}
\newacronym{ebm}{EBM}{Work-Home}
\newacronym{ssim}{SSIM}{Structural Similarity index}
\newacronym{mssim}{MSSIM}{Mean Structural Similarity index}
\newacronym{anpr}{ANPR}{Automatic Number Plate Recognition}

%::::::::: UPORABA KRATICA U TEKSTU :::::::::::::::::
% u tekstu jednostavno na mjestu gdje �elite koristiti odre�enu kraticu, upotrijebite naredbu 
%  \gls{id_kratice}, kao npr. \gls{nfc}
% i u tekstu �e vam automatski biti uba�ena kratica kako je definirana u drugoj zagradi u gornjim definicijama, a ako je u uporabi prvi puta, tada �e prvo biti naveden puni naziv, kako je definiran u tre�oj zagradi, a potom kratica. Za sve ostale slu�ajeve uporabe, bit �e navedena samo kratica.

%:::::::::::::::::::::::::::::::::::::::::::::::::::::
%  podsjetnik nekoliko mogu�ih oblika sintakse
% op�a uporaba: \gls{nfc} % mo�e i za rje�nik i za kratice. Prvi poziv daje dugi i kratki naziv (redoslijedom koji je specificiran u preambuli pomocu \setacronymstyle, a od drugi puta nadalje samo kraticu.
% ako �elite nametnuti ba� neki oblik kori�tenja kratice, imate sljede�e naredbe
%\acrshort{nfc} \\  % samo akronim
%\acrfull{nfc} \\   % akronim i puni naziv
%\acrlong{nfc} \\   % samo puni naziv

%::::::::::::::::::::::::::::::::::::::::::::::::::::
% Kako se generira Pojmovnik u tekstu:
%1. pokrenite LaTeX kompilaciju 1x
%2. u Command Promptu odite radnu mapu gdje su vam datoteke diplomskog rada i utipkajte
%	makeindex -s myDoc.ist -o myDoc.gls   myDoc.glo
%	
%	gdje myDoc zamijenite imenom svoje glavne .tex datoteke (JMBAG_Ime_Prezime.tex)
%3. pokrenite LaTeX kompilaciju jos jednom

%::::::::::::::::::::::::::::::::::::::::::::::::::::