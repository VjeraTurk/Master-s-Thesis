% !TeX encoding = windows-1250
\chapter{Postoje�e metrike za validaciju POM-e}


\section{To�nost POM-e}

To�nost procijenjenih matrica gotovo uvijek se definira u odnosu na referentnu matricu (eng. \textit{grand truth matrix}) koja je dobivena  tradicionalnim postupcima (anketiranje i/ili prebrojavanje vozila). Statisti�ke mjere kvantiziraju sli�nost procijenjenih i �istinitih� vrijednosti, ako su nam one poznate. 

�esto se u literaturi (jednozna�no?) koriste pojmovi \textit{to�nost}, \textit{pouzdanost} i \textit{kvaliteta}. Gotovo uvijek radi se o mjeri koja direktno ima veze sa razina sli�nost odnosno razlikom (gre�ka) s referentnom matricom.

%goodness of fit measure
    
\section{Metrike}

Za procjenu kvalitete matrica dobivenih isklju�ivo anketranjem u radu \cite{Cools:2010.} kori�tena je mjera Mean Apsolute Percentage Error (MAPE), te je dokazano da se zadovoljavaju�a razina kvalitete takvih matrica posti�e tek ako uzorak obuhva�a 50\% populacije. Istaknuta je va�nost kori�tenja dodatnih izvora za izradu matrica.

\cite{Bera:2011.} navodi statisti�ke mjere Relative Error (RE), Total Demand Deviation (TDD), Mean Absolute Error (MAE), Root Mean Square Error (RMSE) te Maximum Possible Relative Error (MPRE) i Travel Demand Scale (TDS)  koji procjenjuju kvalitetu neovisno o referentnoj matrici (no MPRE ne dopu�ta pogre�ke u prebrojavanju prometa, dok TDS ovisi o topologiji mre�e i odabiru ruta). \cite{Djukic:2013.}

U \cite{Frias-Martinez:2012.} kori�ten je \textit{Pearsonov koeficijent korelacije} -  $r$ da bi se utvrdila sli�nost svakog retka matrice dobivene iz CDR s retkom referentne (izlazni tok iz svake polazi�ne �elije). Isti postupak kori�ten je za kontekstualizirane \gls{hw} i \gls{wh} matrice  dobivene iz \gls{cdr} u usporedbi s referentnim matricama dobivenim anketiranjem (to�nost).

Travassoli u svome radu \cite{Travassoli:2016.} navodi nekoliko uobi�ajeno kori�tenih mjera - {$R^{2}$, GEH, \%RMSE te uvodi novu mjeru EBM (temeljenu na svojstvenim vrijednostima matrica) i procjenjuje pouzdanost matrice dobivene iz sustava automatskog prikupljanja podataka u javnom prijevozu (autobus, vlak i trajekt). Spominje i Wasserstein metric, mjeru koja se razlikuje po tome da ne uspore�uje samo vrijednosti parova istih �elija (elementwise). 

Spearmanov koeficijent korelacije ranga kori�ten je u \cite{Graells-Garrido:2016.} za procjenu sli�nosti matrica dobivenih iz CDR sa tada aktualnim matricama dobivenim anketiranjem.


\begin{comment}

mozda po jednu recenicu za svako
\subsubsection{$R^{2}$ }
The R-squared (R2), as one of the most commonly and widely used (Washington et al., 2011), is a statistical measure of how close the data are to the fitted regression line, and itused for comparing between origin-destination pairs of two OD matrices. R2 values rang from 0 to 1, with higher values indicating less difference between OD matrices. 
Along with considering higher value of R2 as a higher level of similarity, the regression line should be close to a 45-degree line through the origin. In this condition, the coefficient of the line should be closer to one and the intercept should be closer to zero. The lower and greater coefficient values indicate the tendency of the pattern to overestimate or underestimate values in the reference OD matrix.\cite{Travassoli:2016.}

\subsubsection{GEH}
Geoffrey E. Havers (GEH) statistic
The GEH statistic is used to evaluate the level of closeness between origin-destination pairs of two OD matrices. The GEH is applied to every pair in the two matrices, with a GEH of less than 5 indicating a good fit (Hollander and Liu 2008). Then, the percentage of OD pairs that have a GEH equal to or less than 5 is calculated to indicate the level of closeness between two OD matrices.
 \cite{Travassoli:2016.}   
 
\subsection{RMSE i \%RMSE} 
The root mean squared error (RMSE) and accordingly the percent root mean squared error(%RMSE) are used to evaluate the closeness of the matrices. The %RMSE is where the variability of the demand is most evident: if two demand matrices were identical, the %RMSE would be equal to zero.
\subsubsection{ MSE, SEM, EBM (SAE)} 
    (How close the models are to the reality)

\subsubsection{Pearson korelacija redova}

\subsubsection{Sperman rank korelacija}

\subsubsection{Wasserstein Metric} (?)

\end{comment}


    
\section{Strukturalna sli�nost}
    
\subsection{MSSI} 
    
\subsubsection{osnovni}\cite{Djukic:2013.}
\subsubsection{pobolj�ani} \cite{Pollard:2013.}\cite{Vuren:2015.}
    
     %TODO: Ovdje si 