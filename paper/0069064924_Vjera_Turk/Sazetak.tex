% !TeX encoding = windows-1250
\vspace{5pt}

%:::::::::::::::::::::::::::::::::::::::::::::::::::::
%:::::::::::: HRVATSKI :::::::::::::::::::::::::::::::
\noindent
Ovo je tekst u kojem se opi�e sa�etak va�ega rada. Tekst treba imati duh rekapitulacije �to je prikazano u radu, nakon �ega slijedi 3-5 klju�nih rije�i (zamijenite dolje postavljene op�enite predlo�ke rije�i nekim suvislim vlastitim klju�nim rije�ima).
%:::::::::::::::::::::::::::::::::::::::::::::::::::::

\vspace{5pt}
%
\noindent \textbf{\textit{Klju�ne rije�i} --- klju�na rije� 1, klju�na rije� 2, klju�na rije� 3} 

%:::::::::::: KRAJ HRVATSKOG DIJELA :::::::::::::::::::


%::::::::::::::::::::::::::::::::::::::::::::::::::::::
%:::::::::::: ENGLESKI ::::::::::::::::::::::::::::::::

%\vspace{-10pt}
\section*{Abstract}
\vspace{-10pt}
This is a text where a brief summary of your work is outlined. The text should have a sense of recap of what was presented in the thesis, followed by 3-5 keywords (replace the general keyword templates below with some meaningful keywords of your own) .
%:::::::::::::::::::::::::::::::::::::::::::::::::::::::

\vspace{5pt}
%
\noindent \textbf{\textit{Keywords} --- keyword 1, keyword 2, keyword 3}

%::::::::::::::::::::::::::::::::::::::::::::::::::::::
%:::::::::::: KRAJ ENGLESKOG DIJELA :::::::::::::::::::

